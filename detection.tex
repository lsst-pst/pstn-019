\subsection{Source Detection}
\label{sec:detection}

TODO: this was previously a \texttt{meas\_algorithms} page; move the bits that are not about detection elsewhere.

The \texttt{meas\_algorithms} package contains a wide variety of astronomical algorithms.
We briefly describe some of them here; for the full list of \texttt{Tasks} defined in this module, see the \href{https://pipelines.lsst.io/v/daily/modules/lsst.meas.algorithms/index.html}{full package documentation}.

\begin{itemize}
    \item \texttt{MeasureApCorrTask} measures aperture corrections on an image (TODO: how? Eli?).
    \item \texttt{NormalizedCalibrationFluxTask} measures SOMETHING TODO: Eli?
    \item \texttt{ObjectSizeStarSelectorTask} is used to find likely PSF-like sources to be used to fit a PSF model during initial calibration.
    \item \texttt{SkyObjectsTask} generates `sky object' \texttt{Footprints} on regions of an image that do not have a \texttt{DETECTED} mask plane set (TODO: link to afw Mask!).
    \item \texttt{SubtractBackgroundTask} fits and subtracts the background of an image, potentially appending it to an earlier fitted background model.
    \item \texttt{ScienceSourceSelectorTask} and \texttt{ReferenceSourceSelectorTask} select sources from a catalog given a set of configurable criteria.
\end{itemize}

This package also contains tools for defining and converting existing third party catalogs to be used as reference catalogs by Science Pipelines code, via \texttt{ConvertReferenceCatalogTask} and its commandline interface \texttt{convertReferenceCatalog}.
These tools are described in more detail in the \href{https://pipelines.lsst.io/v/daily/modules/lsst.meas.algorithms/creating-a-reference-catalog.html}{documentation for creating an LSST reference catalog}.

\subsubsection{SourceDetectionTask}
\label{sec:SourceDetectionTask}

We detect positive and negative sources on an image with \texttt{SourceDetectionTask} to produce a \texttt{SourceCatalog} of \texttt{Footprints}.
This task requires that the image be background subtracted to produce good results.
\texttt{SourceDetectionTask} convolves the image with a Gaussian approximation to the exposure PSF and detects peaks and footprints above a configurable threshold in either signal-to-noise or absolute flux level.
The detected footprints may be significantly blended, depending on the detection threshold and source density in the input image: in order to separate footprints that contain many peaks, some form of deblending (TODO: section link!) must be performed.

\subsubsection{DynamicDetectionTask}
\label{sec:DynamicDetectionTask}

The \texttt{DynamicDetectionTask} is a specialized version of \texttt{SourceDetectionTask} that adapts detection thresholds based on the local background and noise.
This task was initially developed to address detection efficiency issues noted in HSC data.
First, \texttt{DynamicDetectionTask} detects sources using a lower detection threshold than normal.
In so doing, we identify regions of the sky which are unlikely to contain real source flux.
Next, a configurable number of sky objects are placed in these sky regions (1000 by default), and the PSF flux and standard deviation for each of these measurements is calculated.
Using this information, we set the detection threshold such that the standard deviation of the measurements matches the median estimated error.

\subsubsection{Filtering non-astrophysical DiaSources}
\label{MaskStreaksTask}

After the \texttt{detectAndMeasure} step, some bogus DiaSources are expected due to reflective artificial objects orbiting the Earth.
These may originate from, e.g., satellites or space debris, which are most numerous in low-Earth orbit.
Objects in this region, with orbital altitude below 2000 km, move fast across the sky and can leave long streaks that cross one or more LSSTCam detectors.
For more details on how LSSTCam will be affected by bright streaks, see \citet{2020AJ....160..226T,2022A&C....3900584H,2024SPIE13103E..1ZP,2024SPIE13103E..21S,2025arXiv250205418P}.

To reduce the number of DiaSources originating from non-astrophysical sources, we apply a multi-prong approach:

\begin{enumerate}
\item Satellite catalog cross-reference via \texttt{sattle}.
This removes known satellites from the DiaSource catalog and Alerts produced by the AP pipeline.

\item Long trailed source detection and filtering, which is part of \texttt{ap\_association}.
This runs in both the AP and DRP pipelines. It writes detected trailed DiaSources with lengths exceeding about 12 arcsec in a 30 second exposure to a separate catalog, and removes them from subsequent source association.

\item Streak detection and masking.
This is part of \texttt{meas\_algorithms} and runs during \texttt{detectAndMeasure}.
It adds a STREAK mask plane to the difference images by searching for linear features with a Canny filter and the Kernel-Based Hough Transform \citep{2008PatRe..41..299F}.

\item Future: glint detection and masking, which will use a k-d tree range query around DiaSources to identify candidate ``anchor'' sources with enough neighbors that they could be near the center of a glint trail.
When implemented, detected glints will also be masked STREAK.

\item Future: active scheduler avoidance for extremely bright satellites \citep{2022ApJ...941L..15H}, which is not part of the Science Pipelines.
\end{enumerate}

No pixels will be altered or redacted as a result of any of the above techniques.
These approaches are intended to be complementary and enable science users to decide which DiaSources to include in their analyses.
Even together, they are not able to completely eradicate all signatures of non-astrophysical objects from all Science Pipelines difference imaging data products.
We note faint signals of non-astrophysical sources below the detection threshold will remain in the difference images, and these and other effects from the changing population of human-made objects orbiting Earth will affect systematic errors in ways that are impossible to quantify upfront.
For example, \citet{2024ApJ...966L..38T} concluded that most debris smaller than 10 cm in size should not routinely be detected due to being out of focus and moving fast across the focal plane.
Such non-detected objects may as a population increase the background sky brightness during the decade of LSST Operations \citep{2021MNRAS.504L..40K,2023NatAs...7..252B}, which is out of scope for difference imaging.
