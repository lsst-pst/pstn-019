\section{Introduction}
\label{sec:intro}

The NSF-DOE Vera C.\ Rubin Observatory will be performing the 10-year Legacy Survey of Space and Time \citep[LSST;][]{2019ApJ...873..111I} starting in 2025.
Rubin Observatory is located on Cerro Pachon in Chile and consists of the 8.4\,m Simonyi Survey Telescope \citep{2022SPIE12182E..0WT} with the 3.2-gigapixel LSSTCam survey camera \citep{2024SPIE13096E..1SR} performing the main survey and the Rubin Auxiliary Telescope \citep{2020SPIE11452E..0UI} providing supplementary atmospheric calibration data.
The Data Management System \citep[DMS;][]{2022arXiv221113611O} is designed to handle the flow of data from the telescope, approaching 20\,TB per night, in order to issue alerts and to prepare annual data releases.
A central component of the DMS is the LSST Science Pipelines software that provides the algorithms and frameworks required to process the data from the LSST and generate the coadds, difference images, and catalogs to the user community for scientific analysis.

The LSST Science Pipelines software consists of the building blocks and pipeline infrastructure required to construct high performance pipelines to process the data from LSST.
It has been under development since at least 2004 \citep{2004AAS...20510811A} and has evolved significantly over the years as the project transitioned from prototyping \citep{2010SPIE.7740E..15A} and entered into formal construction \citep{2017ASPC..512..279J}.
The software is designed to be usable by other optical telescopes and this has been demonstrated with Hyper Suprime Cam on the Subaru Telescope in Hawaii \citep{2018PASJ...70S...5B} and also with data from the Dark Energy Camera (DECam),  the VISTA infrared camera (VIRCAM), the Wide Field Survey Telescope \citep[WFST;][]{2025arXiv250115018C}, and the Gravitational-wave Optical Transient Observer \citep[GOTO;][]{2021PASA...38....4M}.
It was used to make the first public data release from Rubin Observatory, the LSST Data Preview 1 \citep{RTN-095,10.71929/rubin/2570308}, which consisted of data from the LSST Commissioning Camera \citep[LSSTComCam;][]{10.71929/rubin/2561361,2022SPIE12184E..0JS}.

In this paper we provide an overview of the software system, dividing it into four rough tiers:
\begin{itemize}
\item low-level utility code and our data access, configuration, and execution frameworks, and our core algorithmic primitives are described in sections \ref{sec:support}, \ref{sec:middleware}, and \ref{sec:core}, respectively;
\item reusable mid-level algorithmic components are described in \ref{sec:components};
\item high-level tasks and pipelines are described in \ref{sec:tasks-and-pipelines}, along with some of the details of the algorithms specific to them;
\item analysis and validation tooling is described in \ref{sec:analysis}.
\end{itemize}
There are no sharp boundaries between these tiers; they are better considered to be a loose organizational aid than any kind of formal classification.
We do not include details of the science validation of the individual algorithms, and do not attempt to cover componets of the system in uniform detail; instead we focus attention on those that are unique or novel or not well described elsewhere.
The other components of the LSST DMS, such as the workflow system \citep{2022arXiv221115795G,2024EPJWC.29504026K}, the Qserv database \citep{Wang:2011:QDS:2063348.2063364,C15_adassxxxii} and the Rubin Science Platform \citep{LSE-319,2024ASPC..535..227O}, are not covered in this paper.
