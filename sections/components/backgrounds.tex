\subsection{Background Subtraction}
\label{sec:backgrounds}

Our low-level background subtraction algorithms (in the \texttt{meas\_algorithms} package) operate by binning images, masking out detections and bad pixels, and computing robust statistics (clipped means, by default) on the remaining values in those bins.
The bin averages can then be interpolated via Akima splines or approximated by Chebyshev polynomials.

This code is run and re-run in many different configurations in many different contexts.

\begin{itemize}
\item It is the first step run after Instrument Signature Removal (\S\ref{sec:ip_isr}) on each detector in both the nightly and data release pipelines.
\item In our most advanced configurations of the data release pipelines, we replace these original per-detector backgrounds with new backgrounds fit across the full visit \citep{2019PASJ...71..114A}.
These are configured to subtract the background on large scales, to preserve low surface-brightness features as much as possible.
\item We sometimes run another round of background subtraction on coadds before detecting objects on them, especially when only full-visit backgrounds were subtracted from the input images.
Unfortunately we have found that running background subtraction with a fairly small spatial scale (e.g., 256-pixel bins) -- which intentionally oversubtracts bright stars and low surface-brightness features -- is necessary for the best photometry on most objects, but we prefer to subtract the small-scale background as late as possible in order to allow it to be easily restored for other science cases.
\end{itemize}

In addition, we routinely run background subtraction as a small tweak every time we run detection (\S\ref{sec:detection}).
We \emph{define} our background to be the sum of all light that we do not consider part of a detection, which means it needs to be adjusted when the detection threshold changes (e.g. when we detect fainter objects on coadds).
