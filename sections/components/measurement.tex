\subsection{Source Measurement}
\label{sec:measurement}

After sources are \emph{detected} (\ref{sec:detection}) and optionally \emph{deblended} (\ref{sec:deblending}), the source measurement tasks are responsible for applying a suite of measurement \emph{plugins} on the deblended pixels for each source.
Centroiders, shape measurements, and photometry algorithms are all implemented as measurement plugins.

We also distinguish between measurement on the original detection image (\texttt{SingleFrameMeasurementTask}) vs. measurement on a different image from the original detection (\texttt{ForcedMeasurementTask}).
Measurement could be performed on a single-visit image, a coadd of multiple images, or a difference of images: from the perspective of a measurement plugin, there is no difference between these cases.
\textit{Forced measurement} is performed on one image using a ``reference'' catalog of sources that were detected on another image.

The measurement tasks, plugin base classes, and a suite of standard common plugins are defined in the \texttt{meas\_base} package, including (but not limited to):

\subsubsection{Framework Mechanics}
\label{sec:measurement-interfaces}

Plugins are enabled or disabled in a measurement task via the task's configuration, and each plugin has its own configuration nested within the task configuration.
When a measurement task is constructed, it constructs instances of its enabled plugins, providing them a schema object that they can use to declare and document their output columns.
Each plugin is responsible for defining and filling in columns in the output source catalog, and almost all plugins include columns for uncertainties and at least one flag column to report failures.

Measurement plugins often depend on each other, and must be run in a particular order.
Rather than creating a directed acyclic graph to denote the dependencies, the plugins are batched and are run in any order within a batch.
The batch order is defined by the \texttt{getExecutionOrder} method, with smaller execution numbers being run first.
\texttt{BasePlugin} defines a list of named constants for particular cases:
\begin{enumerate}
    \item \texttt{CENTROID\_ORDER} for plugins that require only footprints and peaks
    \item \texttt{SHAPE\_ORDER} for plugins that require a centroid to have been measured
    \item \texttt{FLUX\_ORDER} for plugins that require both a shape and centroid to have been measured.
\end{enumerate}
The measurement system also provides a \textit{slot} system for predefined aliases to allow a plugin to get a value without knowing exactly what plugin originally computed that value, e.g., \texttt{slot\_Centroid} could point to \texttt{base\_SdssCentroid}, or some other plugin that measures centroids.

While the measurement tasks and plugin interfaces are pure Python, most concrete measurement plugins are implemented in C++, since they need to loop pixels.

When a measurement task is run, it starts by making an empty
\texttt{SourceCatalog} (from the ``afw.table`` library, see \ref{sec:core}) with the plugin-defined schema and one row for each of the \texttt{Footprint} objects returned by previous detection and deblending tasks.
It then temporarily replaces all pixels within \texttt{Footprints} by random noise.
As the task loops over each row in the output catalog, that source's pixels are restored -- either to the original \texttt{Exposure} pixels for isolated or otherwise un-deblended sources, or to the deblender's \texttt{HeavyFootprint} values for deblended children -- and the plugins are called in execution order.
Each plugin is given the full modified \texttt{Exposure} and a row of the output catalog to fill in.
Note that plugins are \emph{not} limited to using only the pixels within a \texttt{Footprint}; they get to decide themselves which pixels to use.
After each source is measured, the task replaces its pixels with noise again, allowing the next source to be measured independently.

\subsubsection{Aperture Corrections}
\label{sec:apcorr}

TODO

% From DP1 release notes:

Rubin processing uses aperture corrections to ensure that different photometry estimators produce consistent results on point sources.
These corrections are measured by applying different algorithms to the same set of bright stars on each single-visit image and interpolating the ratio of each algorithm to a standard one (a background-compensated top-hat aperture flux), which is then used for all photometric calibration.
All fluxes other than the standard algorithm's are then multiplied by the interpolated flux ratio.
Aperture corrections on co-adds are computed by averaging the single-detector ratios with the same weights that were used to combine images.

This scheme has several problems:

\begin{itemize}

    \item These aperture corrections are well-defined for point sources only, but we still apply them for most of our galaxy-focused photometry algorithms (the \texttt{sersic\_*} fluxes are the sole exception), since this at least makes them well-calibrated for poorly-resolved galaxies.

    \item Co-adding apertures with the same weights as the images is only correct in the limit that the images have the same PSF.
    For fixed-aperture photometry a different combination should be used (and will be used in future data releases, if we use this scheme at all), and for PSF-dependent photometry no formally correct combination is possible.

    \item Ratios of fluxes on even bright stars can be very noisy, and in some cases the aperture correction is a significant fraction of our error budget.

\end{itemize}

Improving our approach to aperture corrections is a research project; we are not happy with the current situation, but have not yet identified a satisfactory alternative.


\subsubsection{Sky Objects}

\label{sec:sky-objects}

TODO

\subsubsection{Standard Measurement Plugins}

TODO: highlight important \texttt{meas\_base} algorithms.

\input{sections/components/measurement/gaap}
\subsubsection{Kron Photometry}
\label{sec:kron}

The \texttt{meas\_extensions\_photometryKron} implements Kron photometry \citep{1980ApJS...43..305K}.
Our Kron implementation uses a scaled version of the HSM shapes (See \secref{sec:hsm}) to form an elliptical aperture, and then scales to $2.5$ times the first radial moment.

Our implementation does not correct for the PSF in any way; this means its outputs should only be used for very well-resolved galaxies.
We do not expect our Kron photometry to be competitive with most of our other galaxy photometry algorithms in robustness or precision, but it may be useful for comparison with external measurements.

\subsubsection{HSM Shapes}
\label{sec:hsm}

The \texttt{meas\_extensions\_shapeHSM} package contains the plugins to measure the shapes of objects.
The plugins measure the moments of the sources and PSFs with adaptive Gaussian weights.
The algorithm was initially described in \citet{2003MNRAS.343..459H} and was modified later in \citet{2005MNRAS.361.1287M}.
The implementation of these algorithms lives within the \texttt{hsm} module of the GalSim package \citep{2015A&C....10..121R}.
\texttt{meas\_extensions\_shapeHSM} now interacts directly with the Python layer of GalSim to make the measurements.

The base plugin for measuring moments is the \texttt{HsmMomentsPlugin} and is the parent class of the \texttt{HsmSourceMomentsPlugin} and \texttt{HsmPsfMomentsPlugin} which are specialized to measure on the sources (and objects) and PSFs respectively.
\texttt{HsmSourceMomentsRoundPlugin} is a further specialized plugin that measures the moments with circular Gaussian weights instead of the elliptical ones in \texttt{HsmSourceMomentsPlugin}.
The \texttt{HsmPsfMomentsDebiasedPlugin} adds noise to the PSF image to degrade it to have the same signal-to-noise ratio (SNR) as the source image.
This makes the ellipticity calculated from this plugin have the same bias as the source ellipticity
The PSF moments from this plugin should be used when calculating ellipticity residuals so the bias is largely cancelled.
Having the various specializations as distinct plugins allows an object to be measured under different configurations simultaneously and included in the output catalogs.

In addition to the plugins that measure (adaptive) weighted moments, there are also a series of \texttt{HsmShape} plugins to estimate the PSF-corrected ellipticities of objects.
In particular, the outputs from \texttt{HsmShapeRegaussPlugin} have been used to measure weak gravitational lensing signals in the Hyper Suprime-Cam SSP data \citep{2018PASJ...70S..25M, 2022PASJ...74..421L}.

In addition to the second moments that characterize the size and ellipticity of the PSF, higher-order moments \textemdash\ those beyond second order \textemdash\ capture more subtle aspects of the PSF shape, such as skewness, kurtosis, and other asymmetric or non-Gaussian features.
The \texttt{HigherOrderMomentsPSFConfig} is a plugin within \texttt{meas\_extensions\_shapeHSM} to calculate the higher order moments of the PSF models whereas \texttt{HigherOrderMomentsSourcePlugin} calculates that of the sources (and objects).
The definitions of the higher moments are given in \citet{2023MNRAS.520.2328Z}.
These moments are measured in normalized coordinates, where the normalized $x$-axis is along the major axis and the normalized $y$-axis along the minor.
Such a normalization implies that the moments are dependent only on features of the light profile beyond second moments, and does not scale with the flux, position, size or orientation of the object.
It is designed to be used in conjunction with the \texttt{HsmShapePlugin} and \texttt{HsmPsfMomentsPlugin} plugins, which measure the second moments for the normalization, and provides the HSM adaptive Gaussian kernel.
By default, we compute the third and fourth order moments of the source and PSF images.

\subsubsection{Trailed Sources}
\label{sec:trailed-sources}

Implemented in the \texttt{meas\_extensions\_trailed\-Sources} package, this component measures the properties of trailed sources, such as those caused by moving objects like asteroids or satellites.
It measures the length, angle, flux, centroid, and end points of a trailed source using the \citet{2012PASP..124.1197V} model.
This plugin is designed to refine the measurements of trail length, angle, and end points and of flux and centroid from previous measurement algorithms.

\subsubsection{CModel Galaxy Fitting}
\label{sec:cmodel}

The \texttt{meas\_modelfit} package's CModel algorithm is a reimplemention of the SDSS galaxy-model fitting approach, in which we fit a PSF-convolved elliptical exponential profile and de Vaucouleurs profile to each object separately, and then fit a linear combination of the two with the ellipse parameters held fixed.
This is not as principled as a true bulge-disk decomposition, in which both models are fit simultaneously, but since each fit has fewer degrees of freedom, it can nevertheless work better on low signal-to-noise or poorly-resolved galaxies, where the degeneracies in a bulge-disk decomposition might otherwise lead to completely unphysical parameters.
In fact, in practice, CModel is better though of as a crude single-Sersic fit than a kind of decomposition.

Because these are PSF-convolved models, we expect CModel to provide decent photometry (including for colors) on both small and well-resolved objects.
At present, CModel flux uncertainties are known to be severely underestimated; at least some of this is due to the fact that the ellipse parameter uncertainties cannot easily be propagated into the flux uncertainties in the final fit.

The CModel code is essentially unchanged from the version used in the HSC pipelines; and additional details can be found in \citet{2018PASJ...70S...5B}.

\subsubsection{MultiProfit Galaxy Fitting}
\label{sec:multiprofit}

MultiProFit is a package for Gaussian mixture model fitting \citep{DMTN-312}.
It is primarily used to provide multiband Sersic model fits to objects using all available coadds.
The \texttt{multiprofit} package and its dependencies are included in the science pipelines but can also be installed independently, as they only depend on packages that are also available elsewhere.

The \texttt{meas\_extensions\_multiprofit} package contains pipeline tasks (with interfaces defined in \texttt{pipe\_tasks}) necessary to run \texttt{multiprofit} on coadded and deblended images.
The first of these tasks fits a Gaussian mixture model to the PSF model image at the location of each object in a patch.
This procedure is similar to the shapelet PSF fitting functionality in \texttt{meas\_modelfit} \secrefp{sec:cmodel}.
The main differences are that the components are pure Gaussians (shapelet parameters are not supported), can have independent shapes, and are constrained to have integrals summing to unity (i.e., they are normalized).
Currently, only a maximum of two components are supported; this limitation may be removed in the future.
In all cases, the structural parameters for each component are band-independent, with a separate total flux parameter for each band.
That is, individual components do not have intrinsic color gradients (although the convolved models might, if the PSF parameters vary by band).

\input{sections/components/measurement/reliability}
