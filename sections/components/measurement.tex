\subsection{Source Measurement}
\label{sec:measurement}

After sources are \emph{detected} \secrefp{sec:detection} and optionally \emph{deblended} \secrefp{sec:deblending}, the source measurement tasks are responsible for applying a suite of measurement \emph{plugins} on the deblended pixels for each source.
Centroiders, shape measurements, and photometry algorithms are all implemented as measurement plugins.

We also distinguish between measurement on the original detection image vs. measurement on a different image from the original detection.
Measurement could be performed on a single-visit image, a coadd of multiple images, or a difference of images: from the perspective of a measurement plugin, there is no difference between these cases.
\textit{Forced measurement} is performed on one image using a ``reference'' catalog of sources that were detected on another image.

The measurement tasks, plugin base classes, and a suite of standard common plugins are defined in the \texttt{meas\_base} package, as described in \secref{sec:standard-plugins}.

\subsubsection{Framework Mechanics}
\label{sec:measurement-interfaces}

Plugins are enabled or disabled in a measurement task via the task's configuration (see \secref{sec:appendix_config} for an example), and each plugin has its own configuration nested within the task configuration.
When a measurement task is constructed, it constructs instances of its enabled plugins, providing them a schema object that they can use to declare and document their output columns.
Each plugin is responsible for defining and filling in columns in the output source catalog, and almost all plugins include columns for uncertainties and at least one flag column to report failures.

Measurement plugins often depend on each other, and must be run in a particular order.
Rather than creating a directed acyclic graph to denote the dependencies, the plugins are batched and are run in any order within a batch.
The batch order is defined by the \texttt{getExecutionOrder} method, with smaller execution numbers being run first.
The base class defines a list of named constants for particular cases:
\begin{enumerate}
    \item \texttt{CENTROID\_ORDER} for plugins that require only footprints and peaks
    \item \texttt{SHAPE\_ORDER} for plugins that require a centroid to have been measured
    \item \texttt{FLUX\_ORDER} for plugins that require both a shape and centroid to have been measured.
\end{enumerate}
The measurement system also provides a \textit{slot} system for predefined aliases to allow a plugin to get a value without knowing exactly what plugin originally computed that value, e.g., \texttt{slot\_Centroid} could point to \texttt{base\_SdssCentroid}, or some other plugin that measures centroids.

While the measurement tasks and plugin interfaces are pure Python, most concrete measurement plugins are implemented in C++, since they need to loop over pixels.

When a measurement task is run, it starts by making an empty
\texttt{SourceCatalog} (from the ``afw.table`` library, see \secref{sec:core}) with the plugin-defined schema and one row for each of the \texttt{Footprint} objects returned by previous detection and deblending tasks.
It then temporarily replaces all pixels within \texttt{Footprints} by random noise.
As the task loops over each row in the output catalog, that source's pixels are restored -- either to the original \texttt{Exposure} pixels for isolated or otherwise un-deblended sources, or to the deblender's \texttt{HeavyFootprint} values for deblended children -- and the plugins are called in execution order.
Each plugin is given the full modified \texttt{Exposure} and a row of the output catalog to fill in.
Note that plugins are \emph{not} limited to using only the pixels within a \texttt{Footprint}; they get to decide themselves which pixels to use.
After each source is measured, the task replaces its pixels with noise again, allowing the next source to be measured independently.

\subsubsection{Aperture Corrections}
\label{sec:apcorr}

With many different measures of photometry available for both stars and galaxies, establishing a consistent internal photometric system is a challenge, even before we consider the problem (covered in \secref{sec:calibration}) of mapping that system to absolute photometry via an external reference catalog.

In addition, standard PSF modeling \secrefp{sec:psf_modeling} generally focuses on the core of the true PSF, which is only adequate for photometry for galaxies and fainter stars (while the wings of the PSF matter for bright galaxies as well, the semi-arbitrary definition of the boundary of such galaxies is typically a bigger source of photometry uncertainty).

Our solution to both of these problems is \emph{aperture corrections}, in which we apply multiple photometry algorithms to a suite of isolated bright stars on each detector (by default, the same ones used to build the PSF model), compute the ratio of each algorithm to a standard one (a background-compensated top-hat aperture flux), and then interpolate that ratio using Chebyshev polynomials to other positions on the detector.
The measured fluxes of other objects are then scaled by that interpolated ratio.
This essentially forces all algorithms to produce the same results (on average) on stars.

It is clear that this approach is not exactly correct for galaxies or other extended sources in general, but we believe it is usually better than not correcting the galaxy photometry at all (it is, after all, the right thing to do for barely-resolved galaxies).

This scheme has two other serious limitations.
The first is that many of the measurements we need to correct are noisy, even on bright stars, and the interpolated ratios can add significant uncertainty into the final fluxes.
The second is that aperture corrections cannot in general be coadded along with the images (unlike PSF models); they are not a linear function of the data.
Coadding the ratios \emph{is} a valid first-order approximation in the limit where the photometry does not depend on the PSF and the PSFs of the contributing images are similar, and this is what we do at present, but we cannot currently  quantify the error this approximation introduces.
Our understanding is that using much larger PSF models is the cleanest solution to these problems, but the additional degrees of freedom that would entail may be hard to constrain with the information available.

\subsubsection{Sky Objects}
\label{sec:sky-objects}

Sky objects are a special class of object that are not detected in the image being measured, but rather placed quasi-randomly in regions that are likely to be free of real sources.
They're used to measure the sky background and noise properties of the image, and to assist in setting dynamic detection thresholds (see \secref{sec:DynamicDetectionTask}).
After these pseudo-objects are placed, they're measured and processed in the same way as a true object.
Their measurements are included in output catalogs, with an accompanying flag set to indicate that these are sky objects.

By default, sky objects are placed in regions of the sky whereby no detection footprints or bad pixels are located within an 8-pixel radius.
To generate initial candidate positions, we use the quasi-random Halton sequence number generator with a seed based on the image ID.
This is a deterministic sequence that mimics random sampling, but avoids clumping and gaps that can occur with true random sampling.
Doing so allows us to generate a more unbiased sampling of the entire sky background, often with fewer required sky objects.
The initial list of candidate positions is normally larger than the requested number of sky objects, by default of order 5 times.
Attempts are made at placing sky objects iteratively until the requested number of sky objects is reached or the candidate list is exhausted.
Typically, for visit-level detectors and coadd-level images, we place 100 sky objects.\footnote{When these pseudo-sources are placed into visit-level data, we instead refer to them as sky sources.}
Within the \texttt{DynamicDetectionTask} a larger number of temporary sky objects (default 1000) are placed in order to generate an updated detection threshold.

\subsubsection{Standard Measurement Plugins}
\label{sec:standard-plugins}

The \texttt{meas\_base} package includes a suite of standard measurement plugins.
These include second moment shapes (\texttt{base\_SdssShape}) and adaptive centroids \citep[\texttt{base\_SdssCentroid}]{2003AJ....125.1559P}, circular aperture photometry (\texttt{base\_CircularApertureFlux}), PSF-weighted photometry (\text{base\_PsfFlux}), and many others.
It also includes utility plugins that transform pixel bitmask values into catalog flags (\texttt{base\_PixelFlags}) and apply coordinate transforms to centroids for use in forced photometry.
The algorithms are described in detail in \citet{2018PASJ...70S...5B}.
There are also numerous extension plugins available in other pipeline packages described in the following sections.

\subsubsection{Gaussian Aperture and PSF Photometry}
\label{sec:gaap}

\texttt{meas\_extensions\_gaap} implements the Gaussian Aperture and PSF photometry (GAaP) algorithm \citep{2008A&A...482.1053K}.
It is an aperture photometry algorithm designed to obtain consistent colors of extended objects (i.e., galaxies).
This is done by weighting each (pre-seeing) region of a galaxy by the same pre-defined 2D Gaussian function in all the bands and is thus largely insensitive to the seeing conditions in the different bands.
In practice, this is done by first convolving each object by a kernel (using the same tools described in \secref{sec:dia}) so that the PSF is Gaussian and is larger by about 15\% (this is configurable).
As a second step, each Gaussianized object is then weighted with a Gaussian aperture so that the effective pre-seeing Gaussian aperture is the same for all objects in all the bands.
The plugin is configured to use a series of circular Gaussian apertures and/or an elliptical Gaussian aperture that matches the shape of the object in the reference band.

Although the two-step approach is motivated by the original implementation in \citet{2008A&A...482.1053K}, the implementation of this algorithm within the broader context of the measurement framework makes it different from the implementation used in the Kilo-Degree Survey \citep[KiDS;][]{2024A&A...686A.170W}.
In particular, because neighboring objects are replaced with noise before measurement, Gaussianization of the PSF does not result in increased blending as mentioned in Appendix A2 of \citet{2015MNRAS.454.3500K}.
Furthermore, the uncertainty handling is different.
Correlations in noise introduced due to PSF-Gaussianization is included in the uncertainty estimates.
However, because only per-pixel noise variance is tracked, the noise treatment is forced to assume that the noise is uncorrelated to begin with which is not true on the coadds.
See \citet{DMTN-190} for more details on the implementation details.

Note that this measurements from this plugin do not produce total fluxes, but should only be used to obtain colors.
For total fluxes, measurements from \texttt{cModel} \secrefp{sec:cmodel} or \texttt{MultiProFit} \secrefp{sec:multiprofit} are recommended.

\subsubsection{Kron Photometry}
\label{sec:kron}

The \texttt{meas\_extensions\_photometryKron} implements Kron photometry \citep{1980ApJS...43..305K}.
Our Kron implementation uses a scaled version of the HSM shapes (See \secref{sec:hsm}) to form an elliptical aperture, and then scales to $2.5 \times R_{\mathrm{text}}$, where $R_{\mathrm{text}}$ is the first radial moment.

Our implementation does not correct for the PSF in any way; this means its outputs should only be used for very well-resolved galaxies.
We do not expect our Kron photometry to be competitive with most of our other galaxy photometry algorithms in robustness or precision, but it may be useful for comparison with external measurements.

\subsubsection{HSM Shapes}
\label{sec:hsm}

The \texttt{meas\_extensions\_shapeHSM} package contains the plugins to measure the shapes of objects.
The plugins measure the moments of the sources and PSFs with adaptive Gaussian weights.
The algorithm was initially described in \citet{2003MNRAS.343..459H} and was modified later in \citet{2005MNRAS.361.1287M}.
The implementation of these algorithms lives within the \texttt{hsm} module of the \textsc{GalSim} package \citep{2015A&C....10..121R}.

There are numerous plugins for measuring the moments, for example to use circular Gaussian weights instead of elliptical ones, or to add noise to the PSF image to match the SNR of the source image.
This makes the ellipticity calculated from this plugin have the same bias as the source ellipticity.
The PSF moments from this plugin should be used when calculating ellipticity residuals so the bias is largely cancelled.
Having the various specializations as distinct plugins allows an object to be measured under different configurations simultaneously and included in the output catalogs.

In addition to the plugins that measure (adaptive) weighted moments, there are also a series of plugins to estimate the PSF-corrected ellipticities of objects.
In particular, the outputs from one of these algorithms have been used to measure weak gravitational lensing signals in the Hyper Suprime-Cam SSP data \citep{2018PASJ...70S..25M, 2022PASJ...74..421L}.

In addition to the second moments that characterize the size and ellipticity of the PSF, higher-order moments \textemdash\ those beyond second order \textemdash\ capture more subtle aspects of the PSF shape, such as skewness, kurtosis, and other asymmetric or non-Gaussian features.
The definitions of the higher moments are given in \citet{2023MNRAS.520.2328Z}.

\subsubsection{Trailed Sources}
\label{sec:trailed-sources}

The \texttt{meas\_extensions\_trailed\-Sources} package provides a plugin that measures the properties of trailed sources, such as those caused by moving objects like asteroids or satellites.
It measures the length, angle, flux, centroid, and end points of a trailed source using the \citet{2012PASP..124.1197V} model.
This plugin is designed to refine the measurements of trail length, angle, and end points and of flux and centroid from previous measurement algorithms.

\subsubsection{CModel Galaxy Fitting}
\label{sec:cmodel}

Composite Model (CModel) is a model fitting approach in which a pure exponential and pure de Vaucouleur are each fit separately, and then their linear combination is fit while the ellipse parameters are held fixed.
This model aims to approximate (or model) all of the light coming from objects and is implemented in the \texttt{meas\_modelfit} package.

The CModel approach to model-fit galaxy photometry -- also known as the ``Sloan Swindle''' -- is an approximation to bulge+disk or Sersic model fitting that follows the following sequence:

\begin{itemize}
\item Fit a PSF-convolved elliptical exponential (Sersic n=1) model to the data.
\item Fit a PSF-convolved elliptical de Vaucouleurs (Sersic n=4) model to the data.
\item  Holding the positions and ellipses of both models fixed (only allowing the amplitudes to vary),  fit a linear combination of the two models.
\end{itemize}

 In the limit of pure bulge or pure disk galaxies, this approach yields the same results as a more principled bugle+disk or Sersic fit.
 For galaxies that are a combination of the two components (or have more complicated morphologies, as of course all real galaxies do), it provides a smooth transition between the two models, and the fraction of flux in each of the two parameters is correlated with Sersic index and the true bulge-disk ratio.
 Most importantly, this approach yielded good galaxy colors in the SDSS data processing.

 In this implementation of the CModel algorithm, we actually have 4 stages:

 \begin{enumerate}

 \item In the ``initial''' stage, we fit a very approximate PSF-convolved elliptical model, just to provide a good starting point for the subsequence exponential and de Vaucouleur fits.
 Because we use shapelet/Gaussian approximations to convolved models with the PSF, model evaluation is much faster when only a few Gaussians are used in the approximation, as is done here.
 In the future, we may also use a simpler PSF approximation in the initial fit, but this is not yet implemented.
 We also have not yet researched how best to make use of the initial fit (i.e., how does the initial best-fit radius typically relate to the best-fit exponential radius?), or what convergence criteria should be used in the initial fit.
 Following the initial fit, we also revisit the question of which pixels should be included in the fit.

 \item In the ``exp''' stage, we start with the ``initial'' fit results, and fit an elliptical exponential profile.

 \item In the ``dev'' stage, we start with the ``initial'' fit results, and fit an elliptical de Vaucouleur profile.

 \item Holding the ``exp'' and ``dev'' ellipses fixed, we fit a linear combination of those two profiles.

 \end{enumerate}

 In all of these steps, the centroid is held fixed at a given input value (take from the slot centroid when run by the measurement framework).

TODO

\subsubsection{MultiProfit Galaxy Fitting}
\label{sec:multiprofit}

MultiProFit is a package for Gaussian mixture model fitting \citep{DMTN-312}.
It is primarily used to provide multiband Sersic model fits to objects using all available coadds.
The \texttt{multiprofit} package and its dependencies are included in the science pipelines but can also be installed independently, as they only depend on packages that are also available elsewhere.

The \texttt{meas\_extensions\_multiprofit} package contains pipeline tasks (with interfaces defined in \texttt{pipe\_tasks}) necessary to run \texttt{multiprofit} on coadded and deblended images.
The first of these tasks fits a Gaussian mixture model to the PSF model image at the location of each object in a patch.
This procedure is similar to the shapelet PSF fitting functionality in \texttt{meas\_modelfit} \secrefp{sec:cmodel}.
The main differences are that the components are pure Gaussians (shapelet parameters are not supported), can have independent shapes, and are constrained to have integrals summing to unity (i.e., they are normalized).
Currently, only a maximum of two components are supported; this limitation may be removed in the future.
In all cases, the structural parameters for each component are band-independent, with a separate total flux parameter for each band.
That is, individual components do not have intrinsic color gradients (although the convolved models might, if the PSF parameters vary by band).

