\subsection{Instrument Signature Removal}
\label{sec:isr}

Raw images from charge-coupled devices (CCDs) contain instrumental effects, such as dark currents, tree-rings \citep{2020JATIS...6a1005P}, brighter-fatter \citep{2024PASP..136d5003B}, amplifier offsets, clocking artifacts, or crosstalk between neighboring amplifiers, that can be removed in the data processing.
In the LSST Science Pipelines, this step is called Instrument Signature Removal (ISR) and is the first processing applied to a raw CCD exposure (see, for example, the pipeline workflow example shown in Fig.~\ref{fig:pipe_viz}).
The package that performs ISR on an exposure, called \texttt{ip\_isr}, is a critical component of the pipelines used to process LSST images and requires calibration products produced and verified by the \texttt{cp\_pipe} and \texttt{cp\_verify} packages, as described in \secref{sec:calib_pipe}.
