\subsubsection{CModel Galaxy Fitting}
\label{sec:cmodel}

Composite Model (CModel) is a model fitting approach in which a pure exponential and pure de Vaucouleur are each fit separately, and then their linear combination is fit while the ellipse parameters are held fixed.
This model aims to approximate (or model) all of the light coming from objects and is implemented in the \texttt{meas\_modelfit} package.

The CModel approach to model-fit galaxy photometry -- also known as the ``Sloan Swindle''' -- is an approximation to bulge+disk or Sersic model fitting that follows the following sequence:

\begin{itemize}
\item Fit a PSF-convolved elliptical exponential (Sersic n=1) model to the data.
\item Fit a PSF-convolved elliptical de Vaucouleurs (Sersic n=4) model to the data.
\item  Holding the positions and ellipses of both models fixed (only allowing the amplitudes to vary),  fit a linear combination of the two models.
\end{itemize}

 In the limit of pure bulge or pure disk galaxies, this approach yields the same results as a more principled bugle+disk or Sersic fit.
 For galaxies that are a combination of the two components (or have more complicated morphologies, as of course all real galaxies do), it provides a smooth transition between the two models, and the fraction of flux in each of the two parameters is correlated with Sersic index and the true bulge-disk ratio.
 Most importantly, this approach yielded good galaxy colors in the SDSS data processing.

 In this implementation of the CModel algorithm, we actually have 4 stages:

 \begin{enumerate}

 \item In the ``initial''' stage, we fit a very approximate PSF-convolved elliptical model, just to provide a good starting point for the subsequence exponential and de Vaucouleur fits.
 Because we use shapelet/Gaussian approximations to convolved models with the PSF, model evaluation is much faster when only a few Gaussians are used in the approximation, as is done here.
 In the future, we may also use a simpler PSF approximation in the initial fit, but this is not yet implemented.
 We also have not yet researched how best to make use of the initial fit (i.e., how does the initial best-fit radius typically relate to the best-fit exponential radius?), or what convergence criteria should be used in the initial fit.
 Following the initial fit, we also revisit the question of which pixels should be included in the fit.

 \item In the ``exp''' stage, we start with the ``initial'' fit results, and fit an elliptical exponential profile.

 \item In the ``dev'' stage, we start with the ``initial'' fit results, and fit an elliptical de Vaucouleur profile.

 \item Holding the ``exp'' and ``dev'' ellipses fixed, we fit a linear combination of those two profiles.

 \end{enumerate}

 In all of these steps, the centroid is held fixed at a given input value (take from the slot centroid when run by the measurement framework).

TODO
