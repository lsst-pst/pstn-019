\subsection{Deblending}

\label{sec:deblending}

\emph{Deblending} has become the standard astronomical term for dealing with images where multiple distinct astrophysical sources overlap.
In the LSST Science Pipelines, it specifically means assigning different fractions of the flux of all of the pixels in a \texttt{Footprint} to each of the peaks in that \texttt{Footprint}, which are then each considered a ``child'' source.
This fractional flux assignment is very different from creating a ``segmentation map'' that fully assigns each pixel to each child source, as done by, for example, Source Extractor \citep{1996A&AS..117..393B}.
\texttt{HeavyFootprint} (an extension of \texttt{Footprint} that adds a flattened array of pixel values) is used to pass the per-pixel fluxes to downstream algorithms.

Deblending in the science pipelines is performed differently for single-band (visit) image processing vs.\ multi-band (coadd) image processing.
For single-band images we use a modified version of the SDSS deblender \citep{rhldeblend} from the \texttt{meas\_deblender} package.
For multi-band images, we use a simplified version of the \textsc{Scarlet} deblending algorithm \citep{2018A&C....24..129M} from the \texttt{scarlet\_lite} package.
Our motivation for simplifying the \textsc{Scarlet} algorithm can be found in \citet{DMTN-194}.
More details on the algorithmic implementations of the deblending algorithms are given in \secref{app:deblending}.
