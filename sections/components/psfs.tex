\subsection{PSF Modeling}

Within the pipeline, three distinct PSF models are defined:  \texttt{pcaPsf}, \texttt{PSFex}, and \texttt{Piff}.
Only \texttt{PSFex} and \texttt{Piff} are currently used.
\texttt{PSFex} is a fast, and less accurate PSF estimation and is wrapped within \texttt{meas\_extensions\_psfex}.
In contrast, \texttt{Piff} is a slightly slower, but more accurate PSF estimation that is incorporated in \texttt{meas\_extensions\_piff}. Both \texttt{meas\_extensions\_psfex} and \texttt{meas\_extensions\_piff} are described below.

\subsubsection{meas\_extensions\_psfex}\label{sec:meas_extensions_psfex}

The \texttt{meas\_extensions\_psfex} package provides an interface to a patched version of the PSFEx tool \citep{2011ASPC..442..435B} for modeling spatially varying point spread functions (PSFs).
It uses PSF candidates typically selected from detected sources using a configurable star selector that selects clean, isolated stars.
At its core is \texttt{PsfexPsfDeterminerTask}, which prepares these selected stars for input into PSFEx, runs the external binary, and converts the output into an LSST-specific PSF object (\texttt{PsfexPsf}).
Key parameters such as spatial interpolation order and oversampling ratio are controlled via \texttt{PsfexPsfDeterminerConfig}.

For each CCD in the focal plane, PSFEx independently models the PSF as a linear combination of basis vectors and captures spatial variation using polynomial interpolation.
\texttt{PsfexStarSelectorTask} offers a built-in mechanism for star selection using strict cuts on signal-to-noise ratio, FWHM range, ellipticity, and quality flags.
The package raises specific exceptions for common failure modes: when no stars are available, when none pass quality cuts, or when too few good stars remain to support the required model complexity--as determined by the degrees of freedom needed for the fit.
In the latter case, the insufficient sample size forces the PSFEx model to reduce its polynomial degree to an unsupported level.
The code also includes hooks for visual debugging (via \texttt{afwDisplay} and optional \texttt{matplotlib} plots), enabling visual inspection of PSF candidates and rejection reasons during development or QA.

\texttt{meas\_extensions\_psfex} is used in contexts where PSFEx compatibility is required or when modeling speed is preferred over robustness.
Another external PSF modeler (discussed in \S\ref{sec:meas_extensions_piff}) provides greater robustness but is slower in comparison.

\subsubsection{meas\_extensions\_piff}
\label{sec:meas_extensions_piff}

The \texttt{meas\_extensions\_piff} package is a wrapper around the PSF package \texttt{Piff} used to estimate and compute the PSF \citep{2021ascl.soft02024J,2021MNRAS.501.1282J}.
\texttt{Piff} is a modular package that supports various PSF models, interpolation schemes, coordinate systems, and can operate on
a per-CCD basis or over the full field of view, as indicated by its name.
The implementation within  \texttt{meas\_extensions\_piff} does not exploit the full modularity of \texttt{Piff}; instead, it closely follows the method used for cosmic shear analysis like in DES \citep{2021MNRAS.501.1282J,2025OJAp....8E..26S}.

The PSF model utilized is a \texttt{PixelGrid}, and the interpolation is performed using \texttt{BasisPolynomial} interpolation \citep{2021MNRAS.501.1282J}.
Modeling is executed per CCD and can employ either pixel or sky coordinates.
A key difference from \texttt{PSFex} is that  \texttt{Piff} implements outlier rejection based on chi-squared criteria \citep[see][for more details]{2021MNRAS.501.1282J}.

Most of the configuration described here is adjustable through the \texttt{PiffPsfDeterminerConfig} that are exposing some of the configurable parameters of \texttt{Piff} and can be fine-tuned for a dedicated survey.
However, some important features that were implemented by \citet{2021MNRAS.501.1282J} and \citet{2025OJAp....8E..26S} have not yet been enabled but will be available in the near future.
While \citet{2021MNRAS.501.1282J} operates in sky coordinates with a WCS that includes CCD distortions such as treerings,  \texttt{meas\_extensions\_piff} can work in sky coordinates and incorporate WCS; as written, it does not, however, account for CCD distortions like tree rings.
Additionally, although \citet{2025OJAp....8E..26S} incorporated a color correction to account for chromatic effects on the PSF, this correction has not yet been implemented in  \texttt{meas\_extensions\_piff}.
