\subsection{Single-Frame Processing and Calibration}

\subsubsection{CalibrateImage}
\label{sec:CalibrateImage}

\texttt{CalibrateImageTask}, from the \texttt{pipe\_tasks} package, performs ``single frame processing'' on a post-ISR (\S\ref{sec:isr}) single detector exposure.
We repair and mask cosmic rays and defects, perform an initial set of detection (\S\ref{sec:detection}) and measurement (\S\ref{sec:measurement}) passes to estimate the image Point Spread Function (PSF), compute an astrometric (\S\ref{sec:astrometryTask}) and photometric (\S\ref{sec:photoCal}) calibration, and compute summary statistics on the resulting nanojansky-calibrated exposure and catalog.
The primary user-facing outputs of this task are the photometrically calibrated, background-subtracted \texttt{preliminary\_\-visit\_\-image}, the calibrated \texttt{preliminary\_\-visit\_\-image\_\-background} that was subtracted from it, and \texttt{single\_\-visit\_\-star\_\-unstandardized}, a catalog of bright point-like sources that were used as inputs to calibration, with only a small number of measurements performed on them.
As this task only processes a single detector, it is used by both DRP (\S\ref{sec:drp_pipe}) and AP (\S\ref{sec:ap_pipe}), though with different configurations (AP is focused on latency, while DRP performs more measurements).
For more details of the exact steps performed in this task, see the \href{https://pipelines.lsst.io/v/daily/modules/lsst.pipe.tasks/tasks/lsst.pipe.tasks.calibrateImage.CalibrateImageTask.html}{\texttt{Calibrate\-Image\-Task} pipelines documentation}.

\subsubsection{ReprocessVisitImage}
\label{sec:ReprocessVisitImage}

In DRP, \texttt{ReprocessVisitImageTask}, from the \texttt{drp\_tasks} package, takes the outputs of the global astrometric and photometric models, a visit-level background, and PSF model and re-runs detection and measurement on the post-ISR single-detector exposure.
The primary user-facing outputs of this task are the photometrically calibrated, background-subtracted \texttt{visit\_image}, the calibrated \texttt{visit\_image\_background} that was subtracted from it, and \texttt{source\_unstandardized}, a catalog of all sources detected to 5-sigma, with all relevant measurements performed on them (this is standardized and consolidated into the \texttt{source} per-visit catalog).
Besides its primary purpose of performing detection and measurement using the ``best'' available inputs, this task also allows us to only have to save a small number of relatively small-sized intermediate products in order to re-generate the \texttt{visit\_image} from a \texttt{raw} image, reducing long-term storage needs.
For more details of the exact steps performed in this task, see the \href{https://pipelines.lsst.io/v/daily/modules/lsst.drp.tasks/tasks/lsst.drp.tasks.reprocess_visit_image.ReprocessVisitImageTask.html}{ReprocessVisitImage pipelines documentation}.
