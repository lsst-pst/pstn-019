\subsubsection{\texttt{drp\_tasks}}

\textbf{Coaddition Tasks}
\label{sec:coaddition-tasks}

The LSST Science Pipelines provide a modular task framework for constructing coadds from multiple single-epoch images.
Coadds are used as static-sky templates for image subtraction and detecting and measuring faint sources.
The coaddition process is divided into two main stages: resampling the input images onto a common projection and stacking those resampled images into a single coadd.
Each stage is implemented via configurable tasks that allow the pipelines to be adapted for different instruments and observing strategies.
The first step in coaddition is to resample each single-epoch exposure onto a common projection and pixel grid called a \textit{skyMap}.
This step is performed by the following Tasks:
\begin{itemize}
\item \texttt{MakeDirectWarpTask} performs a straightforward resampling of calibrated exposures.
 \item \texttt{MakePsfMatchedWarpTask} also convolves them to a configurable, common model-PSF.
 This PSF-homogenized variant is useful when the scientific goals require uniform PSF properties across the coadd and is used for artifact rejection during coaddition
 \end{itemize}

All warping tasks use a configurable interpolation kernel. A 5th-order Lanczos kernel is used by default, balancing fidelity and computational efficiency.
Input images are geometrically transformed using the World Coordinate System (WCS) and interpolated onto the target projection defined by a tract and patch geometry.
Each resulting resampled image is called a \texttt{warp}.

Once warps are generated, they are stacked into a final coadd by \texttt{AssembleCoaddTask} or one of its subclasses.
The default implementation, \texttt{CompareWarpAssembleCoaddTask}, performs outlier rejection to remove transient artifacts such as cosmic rays, satellite trails, and moving objects.
The algorithm compares pixel values across epochs and masks those that significantly deviate from the expected distribution.
The artifact rejection algorithm is detailed in \citet{DMTN-080}.
By default, weights for stacking are derived from the inverse of the average variance of each warp, with optional filters on PSF quality and seeing.
The stacked image is accompanied by a mask plane and variance map, and the set of input PSF models is combined into a spatially-varying coadd PSF model (\texttt{CoaddPsf}) to serve as the PSF model for the coadd.

Users can adapt the pipeline to their data by modifying warp selection criteria, choosing a projection as defined in a \texttt{SkyMap} object, and adjusting artifact rejection parameters.
While the default LSST pipelines build direct (non-PSF-matched) coadds for source detection and measurement, the modularity of the task framework makes it easy to substitute PSF-homogenized coadds or experiment with alternative stacking strategies \citep{DMTN-015}.
