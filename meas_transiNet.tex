\subsection{Reliability Scoring}

The \texttt{meas\_transiNet} package determines a numerical score for input cutout images using pre-trained machine-learning models.
Image differencing may produce false detections, so time-domain surveys chacteristically use machine learning classifiers to distinguish astrophysical sources from artifacts \citep[``Real/Bogus;'' e.g.,][]{2012PASP..124.1175B, 2015AJ....150...82G, 2019MNRAS.489.3582D}.

The \texttt{meas\_transiNet} defines ``model packages'' that consist of a python architecture class, a PyTorch \citep{NEURIPS2019_bdbca288} weights file, and associated metadata.
The inference task may be configured to load a model package from disk or from the Butler.

The \texttt{RBTransiNetTask} PipelineTask takes as input three square cutouts of configurable size from the science, template, and difference images centered on the location of a source.
These images are concatenated, batched into Torch blobs, and passed to the model for inference.
Either CPU or GPU backends may be used for inference.
The output of the task is a single float ranging from 0--1 for each cutout triplet, with higher values indicating that the DIASource is more likely to be astrophysical.
These reliability scores are then joined with the DIASource catalogs by a later transformation task.
Detailed discussion of the model architecture, training, and performance will be presented in T.\ Acero Cuellar et.\ al (in prep.).
