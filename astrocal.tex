\subsection{Astrometric Calibration}

\subsection{meas\_astrom}

Single frame astrometric fits are performed by \texttt{AstrometryTask} in \texttt{meas\_astrom} and run in \texttt{CalibrateImage} (TODO: crosslink?).
This task matches a catalog of sources detected and measured on an image to a reference catalog and solves for the \textit{World Coordinate System} (WCS) of the image.
Matching and WCS fitting are performed iteratively, to reject astrometric outliers.
The matcher is either the optimistic (\texttt{MatchOptimisticBTask}) or pessimistic (\texttt{MatchPessimisticBTask}) matcher from V. Tabur 2007 (TOOD: citep!), with the pessimistic matcher used by default due to better performance on dense fields; see \citep{DMTN-031} for details.
The WCS fitter can be a simple affine model on top of the known camera geometry (TODO: link to afw cameraGeom!), as in \texttt{FitAffineWcsTask}, or a FITS TAN-SIP WCS (TODO: link to FITS WCS paper!), as in \texttt{FitTanSipWCSTask} or \texttt{FitSipDistortionTask}.
We default to fitting the simple affine model because we have a well fit distortion model from running \texttt{gbdes} in DRP \ref{sec:gbdes}, thus we do not need the extra degrees of freedom provided by a TAN-SIP model.

Because \texttt{AstrometryTask} only requires a single image, it is suitable for use during Alert Production, which does not have access to the entire focal plane.
This single frame astrometric fit is sufficient for initial calibration and difference imaging (assuming the modeled camera geometry is a close match to the true camera distortions), but during DRP we perform a full focal plane fit with \texttt{gbdes}.

\subsection{gbdes}
\label{sec:gbdes}

\texttt{gbdes} \citep{2022ascl.soft10011B,2017PASP..129g4503B}
