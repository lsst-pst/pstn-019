\section{Detection and measurement}
\label{sec:meas}

We perform detection and measurement on images with the \texttt{meas} framework.
We distinguish between \textit{detection} and \textit{measurement}:
\begin{itemize}
    \item \textit{detection}: identifying \textit{Footprints} (TODO: add afw link!) of sources as being above a given flux or signal-to-noise level (see \ref{sec:SourceDetectionTask}).
    \item \textit{measurement}: running \texttt{plugins} on each source in the image to compute properties of that source (e.g. a centroid or aperture flux) (see below).
\end{itemize}

We also distinguish between measurement on the original detection image (\textit{single-frame measurement}) vs. measurement on a different image from the original detection (\textit{forced measurement}).
Measurement could be performed on a single raw or calibrated image, a coadd of multiple images, or a difference of images: from the perspective of a measurement plugin, there is no difference between these cases.
\textit{forced measurement} is performed on one image, using a "reference" catalog of sources that were detected on another image.

\subsection{meas\_base}
\label{sec:meas_base}

The \texttt{meas} framework interface is defined in the \texttt{meas\_base} package.
Measurement plugins have the \texttt{Plugin} suffix if they are defined in python, and the \texttt{Algorithm} suffix if they are defined in C++.
This package defines base classes for plugins (\texttt{SingleFramePlugin}, \texttt{ForcedPlugin} in python; \texttt{SingleFrameAlgorithm}, \texttt{ForcedAlgorithm} in C++) and the measurement tasks that can be configured to run them (\texttt{SingleFrameMeasurementTask}, \texttt{ForcedMeasurementTask}, \texttt{CatalogCalculationTask}), as well as some concrete implementations of plugins (\texttt{ApertureFluxAlgorithm}, \texttt{BlendednessAlgorithm}, \texttt{CircularApertureFluxAlgorithm}, \texttt{GaussianFluxAlgorithm}, \texttt{LocalBackgroundAlgorithm}, \texttt{PeakLikelihoodFluxAlgorithm}, \texttt{PixelFlagsAlgorithm}, \texttt{PsfFluxAlgorithm}, \texttt{ScaledApertureFluxAlgorithm}, \texttt{SdssCentroidAlgorithm}, \texttt{SdssShapeAlgorithm}).
Each plugin has an associated config class, suffixed with \texttt{Config} in python or \texttt{Control} in C++ (e.g. \texttt{SdssCentroidAlgorithm} has \texttt{SdssCentroidControl}), used to configure parameters of that specific algorithm.

\subsubsection{Measurement plugins}
\label{sec:plugins}

Plugins are added to a \textit{registry}, so that they and their outputs can be referred to by a shorter common name that identifies the package it was defined in, for example \texttt{lsst.meas.base.SdssCentroidAlgorithm} is registered as \texttt{base\_SdssCentroid}.
This way, measurements produced by each plugin will have consistent, distinct names in the output schema, e.g. \texttt{base\_SdssCentroid\_x}, \texttt{base\_SdssCentroid\_y}, \texttt{base\_SdssCentroid\_flag}.

Measurement plugins often depend on each other, and must be run in a particular order.
Rather than creating a directed acyclic graph to denote the dependencies, the plugins are batched and and are run in any order within a batch.
The batch order is defined by the \texttt{getExecutionOrder} method, with smaller execution numbers being run first.
\texttt{BasePlugin} defines a list of named constants for particular cases:
\begin{enumerate}
    \item \texttt{CENTROID\_ORDER} for plugins that require only footprints and peaks
    \item \texttt{SHAPE\_ORDER} for plugins that require a centroid to have been measured
    \item \texttt{FLUX\_ORDER} for plugins that require both a shape and centroid to have been measured.
\end{enumerate}

Measurement plugins output their results to a \texttt{SourceCatalog} (TODO: crosslink to afw section!), which has a \textit{slot} system for predefined aliases to allow a plugin to get a value without knowing exactly what plugin originally computed that value, e.g. \texttt{slot\_Centroid} could point to \texttt{base\_SdssCentroid}, or some other plugin that measures centroids.

\subsubsection{SingleFrameMeasurementTask}
\label{sec:SingleFrameMeasurementTask}

Single frame measurement requires a catalog of detected source \texttt{Footprints}, which could still be blended, or could have been deblended (TODO: crosslink?).
When initialized, the task creates a schema from the configured plugins, which defines the contents of the output catalog and cannot be modified after initialization.

Before performing any measurement, this task replaces all sources with noise (via the \texttt{NoiseReplacer}) in the regions defined by their detected \texttt{Footprints}.
The task then loops over all "parent" sources (those that were not deblended and those that represent the un-deblended state of blends), and then loops over all "children" of parents (if any).
For each such source, the source footprint is re-inserted into the image, all measurement plugins are run, and the footprint is then replaced with noise again.
Then, for blended sources, the parent is inserted, measured (running plugins on both the parent and jointly on all the children via \texttt{measureN}), and again removed.

\subsubsection{ForcedMeasurementTask}
\label{sec:ForcedMeasurementTask}

Forced measurement uses the known pixel position of objects from a reference catalog to constrain measurements on another image.
Typically only photometric measurements are scientifically useful, as the centroid and shape are defined by the reference catalog, and transformed to the coordinate system of the image being measured on (e.g. shifting to the appropriate x/y origin, or transforming through the respective WCSs).
Other than this coordinate transformation, forced measurement proceeds much like single frame measurement above.
Two concrete implementations of the task include \texttt{ForcedPhotCcdTask} for single-visit images and \texttt{ForcedPhotCoaddTask} for coadd patch images, both using the output of a previous single frame measurement run on coadds as the reference catalog.

\subsection{meas\_algorithms}
\label{sec:meas_algorithms}

The \texttt{meas\_algorithms} package contains a wide variety of astronomical algorithms.
We briefly describe some of them here; for the full list of \texttt{Tasks} defined in this module, see the \href{https://pipelines.lsst.io/v/daily/modules/lsst.meas.algorithms/index.html}{full package documentation}.

\begin{itemize}
    \item \texttt{MeasureApCorrTask} measures aperture corrections on an image (TODO: how? Eli?).
    \item \texttt{NormalizedCalibrationFluxTask} measures SOMETHING TODO: Eli?
    \item \texttt{ObjectSizeStarSelectorTask} is used to find likely PSF-like sources to be used to fit a PSF model during initial calibration.
    \item \texttt{SkyObjectsTask} generates \texttt{Footprints} on regions of an image that do not have a \texttt{DETECTED} mask plane set (TODO: link to afw Mask!).
    \item \texttt{SubtractBackgroundTask} fits and subtracts the background of an image, potentially appending it to an earlier fitted background model.
    \item \texttt{ScienceSourceSelectorTask} and \texttt{ReferenceSourceSelectorTask} select sources from a catalog given a set of configurable criteria.
\end{itemize}

This package also contains tools for defining and converting existing third party catalogs to be used as reference catalogs by Science Pipelines code, via \texttt{ConvertReferenceCatalogTask} and its commandline interface \texttt{convertReferenceCatalog}.
These tools are described in more detail in the \href{https://pipelines.lsst.io/v/daily/modules/lsst.meas.algorithms/creating-a-reference-catalog.html}{documentation for creating an LSST reference catalog}.

\subsubsection{SourceDetectionTask}
\label{sec:SourceDetectionTask}

We detect positive and negative sources on an image with \texttt{SourceDetectionTask} to produce a \texttt{SourceCatalog} of \texttt{Footprints}.
This task requires that the image be background subtracted to produce good results.
\texttt{SourceDetectionTask} convolves the image with a Gaussian approximation to the exposure PSF and detects peaks and footprints above a configurable threshold in either signal-to-noise or absolute flux level.
The detected footprints may be significantly blended, depending on the detection threshold and source density in the input image: in order to separate footprints that contain many peaks, some form of deblending (TODO: section link!) must be performed.

\subsubsection{DynamicDetectionTask}
\label{sec:DynamicDetectionTask}

TODO: someone else will have to write this.

\subsubsection{MaskStreaksTask}
\label{MaskStreaksTask}

TODO: for Meredith or Clare?

\subsection{Deblending \footnote{This section assumes that a dection section has already been written. Some changes might be necessary if there are topics not covered in the detection section or topics that are duplicated in this section."}}

\label{sec:deblending}

Deblending in the science pipelines is performed differently for single-band (visit) image processing vs. multi-band (coadd) image processing.
This section gives a basic description of each algorithm.

\subsubsection{Single-band Deblending}

\label{sec:singleband_deblending}

Deblending on single-band images (ie. visit) is performed using the \texttt{meas\_deblender} package and is based on the deblender used in SDSS (citation needed), with a few differences that will be discussed shortly.
Similar to the SDSS deblender, the LSST deblender creates a template for each source in a blend using a very simple (yet computationally efficient) model for each peak position in a parent Footprint.
Once a template has been created for each peak in the blend, the deblender combines all of the source templates into a single blend model by summing their values in each pixel.
For each pixel in a source template, the ratio of the source template value to the total blend model is calculated and used to weight the pixel value from the image to create a model for each source.
The source models are thus flux conserving in that adding them together will yield the original image except for pixels that do not appear in any of the individual templates.
A cleanup algorithm is then run to allocate the remaining pixels to one of the sources in the blend based on a set of criteria including distance to the center, brighness of the nearest sources, etc.

\subsubsection{Deblender Template Generation}

The main ansatz of the SDSS deblending algorithm is that the flux from stars and galaxies in a ground based telescope is nearly 180 degree symmetric.
Figure \ref{fig:simple_blend} illustrates how a 1D slice through the center of two blended sources can exploit this symmetry by setting the pixels on opposite sides of the (integer) center pixel to the minimum value of both pixels.
In other words, for simple blends of only two sources the deblender can use the flux on the non-blended side to constrain the value of the flux on the blended side.
Despite the fact that stars (PSFs) and galaxies are not exactly symmetric, especially since their position is not exactly centered in the center of a single pixel, this algorithm works quite well for generating templates in simple blends that very nearly approximate each source when redistributing flux from the image.

For sources with low SNR the algorithm fails due to noise in the image, generating galaxy templates that are typically very jagged and unphysical.
To combat this, for each \texttt{peak} in the parent \texttt{Footprint} the deblender first attempts to fit the flux from the image with a simple PSF model that allows its position, amplitude, and a linear background, to vary.
If the fit has a reasonable $\chi^2$ value then the deblender will use this scaled PSF model as a template for the source.
Only for sources that cannot be adequately modeled with the PSF are the symmetric templates used.

The main failure point of this algorithm is when three (or more) sources lie along the same axis.
For example, Figure \ref{fig:complex_blend} illustrates a 1D slice through the center of three aligned sources.
In this case the minimum pixel on each side of the central source cannot constrain the flux at that radial location and results in a template that has extra bumps from its neighbors.
This turns out to be more catastrophic than one might expect.
Notice that even the neighboring sources, which have very good templates created by using symmetry on their unblended side, have their resulting models contaminated due to the central object that steals flux from both of them.
In single visits the number of "three in a row" blends is small enough that we sacrifice the quality of the models for efficiency and still use the single-band deblender.
For LSST-depth coadds this becomes a significant problem, as deep coadds can have as much as 40\% of blends having 3 or more sources and a more sophisticated algorithm is needed.

\begin{figure}[htb]
    \centering
    \includegraphics[width=0.48\textwidth]{simple_1D_blend.pdf}
    \caption{A 1D slice of two blended Gaussian sources illustrating how symmetry can be utilized to model blended sources.}
    \label{fig:simple_blend}
\end{figure}

\begin{figure}[htb]
    \centering
    \includegraphics[width=0.48\textwidth]{complex_1D_blend.pdf}
    \caption{A 1D slice through three aligned Gaussian sources, demonstrating a failure case of relying on symmetry for generating deblender templates. Notice that for sources 2 and 3 the templates are reasonable but due to the inability of source 1 to use symmetry to constrain flux in the blended region, the resulting models for all three sources are poor. This catastrophic "three in a row" problem was part of the motivation for creating \texttt{scarlet} to incorporate spectral information and a more rigorous iterative deblending algorithm.}
    \label{fig:complex_blend}
\end{figure}

\subsubsection{Multi-band Deblending}
\label{sec:multiband_deblending}

The multi-band deblender is an implementation of the scarlet deblending algorithm described in \cite{2018ascl.soft03003M}.
In our implementation, \texttt{scarlet\_lite}, we have made our own set of simplifying assumptions that are different from the original scarlet algorithm to make it more efficient when used in a large ground based survey like LSST.
Similar to the original \texttt{scarlet} we make the assumption that astrophysical objects can be thought of as a collection of components, where each component has the properties
\begin{itemize}
    \item Components have a single color (spectrum) that is the same in all pixels over its shape (morphology)
    \item Components have flux that monotonically decreases from the center
    \item Component flux is additive
\end{itemize}

The classic example is decomposing a single galaxy into bulge and disk components, where both the bulge and disk share a common center but have different spectra and morphologies.
Something more complex, like a grand design spiral, could in theory be modeled as a source with multiple components, where spiral arms and star forming regions could still be thought of as separate monotonic components.
For the science pipelines we ignore those more complicated structures, as detection typically already shreds large galaxies into multiple sources.
Instead we use a signal to noise cut where low flux sources are modeled with a single component and higher flux sources are modeled with two components.

Scarlet lite initializes models with nearly the same templates as those generated by the single-band deblender.
Using a $\chi^2$-like monochromatic image created by weighting each band by its inverse variance, scarlet lite creates initial morphology models that are symmetric from the center in the monochromatic image, with the additional constraint that the flux is monotonically decreasing from the center.
In order to satisfy the constraint that all pixels in the morphology have the same spectrum, scarlet models exist in a partially deconvolved frame with the seeing of a well sampled but narrow Gaussian.
The initial spectrum of each source is determined using a least squares fit of each monochromatic morphology, convolved with the difference kernel in each band to match the image, for each component.
It then uses proximal-ADAM (PADAM: \cite{Melchior2019}) to iteratively update the spectrum and morphology with the given constraints until convergence or a maximum number of iterations is reached.
It should be noted that although we do use symmetry to initialize the scarlet models, we do not implement a symmetry constraint and the final models are not guaranteed to be symmetric.
The models are stored as the \texttt{deepCoadd\_scarletModelData} data product, which contains all of the blends for a single patch.
Like the single-band deblender, the \texttt{scarlet\_lite} models are only used as templates to redistribute flux from the image and all measurements are performed on the flux redistributed models.

\subsection{meas\_extensions\_convolved}

\subsection{meas\_extensions\_gaap}

\subsection{meas\_extensions\_photometryKron}

\subsection{PSF Modeling}

Within the pipeline, three distinct PSF models are defined:  \texttt{pcaPsf}, \texttt{PSFex}, and \texttt{Piff}.
Only \texttt{PSFex} and \texttt{Piff} are currently used.
\texttt{PSFex} is a fast, and less accurate PSF estimation and is wrapped within \texttt{meas\_extensions\_psfex}.
In contrast, \texttt{Piff} is a slightly slower, but more accurate PSF estimation that is incorporated in \texttt{meas\_extensions\_piff}. Both \texttt{meas\_extensions\_psfex} and \texttt{meas\_extensions\_piff} are described below.


\subsection{meas\_extensions\_psfex}

\subsubsection{meas\_extensions\_piff}


The \texttt{meas\_extensions\_piff} package is a wrapper around the PSF package \texttt{Piff} used to estimate and compute the PSF \citep{2021ascl.soft02024J,2021MNRAS.501.1282J}.
\texttt{Piff} is a modular package that supports various PSF models, interpolation schemes, coordinate systems, and can operate on
a per-CCD basis or over the full field of view, as indicated by its name.
The implementation within  \texttt{meas\_extensions\_piff} does not exploit the full modularity of \texttt{Piff}; instead, it closely follows the method used for cosmic shear analysis like in DES \citep{2021MNRAS.501.1282J,2025OJAp....8E..26S}.

The PSF model utilized is a \texttt{PixelGrid}, and the interpolation is performed using \texttt{BasisPolynomial} interpolation \citep{2021MNRAS.501.1282J}.
Modeling is executed per CCD and can employ either pixel or sky coordinates.
A key difference from \texttt{PSFex} is that  \texttt{Piff} implements outlier rejection based on chi-squared criteria \citep[see][for more details]{2021MNRAS.501.1282J}.

Most of the configuration described here is adjustable through the \texttt{PiffPsfDeterminerConfig} that are exposing some of the configurable parameters of \texttt{Piff} and can be fine-tuned for a dedicated survey.
However, some important features that were implemented by \citet{2021MNRAS.501.1282J} and \citet{2025OJAp....8E..26S} have not yet been enabled but will be available in the near future.
While \citet{2021MNRAS.501.1282J} operates in sky coordinates with a WCS that includes CCD distortions such as treerings,  \texttt{meas\_extensions\_piff} can work in sky coordinates and incorporate WCS; as written, it does not, however, account for CCD distortions like tree rings.
Additionally, although \citet{2025OJAp....8E..26S} incorporated a color correction to account for chromatic effects on the PSF, this correction has not yet been implemented in  \texttt{meas\_extensions\_piff}.


\input{meas_shapeHSM}
\subsection{meas\_extensions\_simpleShape}

\subsection{meas\_extensions\_trailedSources}

\input{meas_modelfit}
\subsection{meas\_extensions\_multiprofit}
\label{sec:meas_extensions_multiprofit}

\subsection{Reliability Scoring}
\label{sec:reliability}

The \texttt{meas\_transiNet} package determines a numerical score for input cutout images using pre-trained machine-learning models.
Image differencing may produce false detections, so time-domain surveys chacteristically use machine learning classifiers to distinguish astrophysical sources from artifacts \citep[``Real/Bogus;'' e.g.,][]{2012PASP..124.1175B, 2015AJ....150...82G, 2019MNRAS.489.3582D}.

The \texttt{meas\_transiNet} defines ``model packages'' that consist of a python architecture class, a PyTorch \citep{NEURIPS2019_bdbca288} weights file, and associated metadata.
The inference task may be configured to load a model package from disk or from the Butler.

The \texttt{RBTransiNetTask} PipelineTask takes as input three square cutouts of configurable size from the science, template, and difference images centered on the location of a source.
These images are concatenated, batched into Torch blobs, and passed to the model for inference.
Either CPU or GPU backends may be used for inference.
The output of the task is a single float ranging from 0--1 for each cutout triplet, with higher values indicating that the DIASource is more likely to be astrophysical.
These reliability scores are then joined with the DIASource catalogs by a later transformation task.
Detailed discussion of the model architecture, training, and performance will be presented in T.\ Acero Cuellar et.\ al (in prep.).

