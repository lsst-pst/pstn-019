\section{Detection and measurement}
\label{sec:meas}

We perform detection and measurement on images with the \texttt{meas} framework.
We distinguish between \textit{detection} and \textit{measurement}:
\begin{itemize}
    \item \textit{detection}: identifying \textit{Footprints} (TODO: add afw link!) of sources as being above a given flux or signal-to-noise level (see \ref{sec:SourceDetectionTask}).
    \item \textit{measurement}: running \texttt{plugins} on each source in the image to compute properties of that source (e.g. a centroid or aperture flux) (see below).
\end{itemize}

We also distinguish between measurement on the original detection image (\textit{single-frame measurement}) vs. measurement on a different image from the original detection (\textit{forced measurement}).
Measurement could be performed on a single raw or calibrated image, a coadd of multiple images, or a difference of images: from the perspective of a measurement plugin, there is no difference between these cases.
\textit{forced measurement} is performed on one image, using a "reference" catalog of sources that were detected on another image.

\subsection{meas\_base}
\label{sec:meas_base}

The \texttt{meas} framework interface is defined in the \texttt{meas\_base} package.
Measurement plugins have the \texttt{Plugin} suffix if they are defined in python, and the \texttt{Algorithm} suffix if they are defined in C++.
This package defines base classes for plugins (\texttt{SingleFramePlugin}, \texttt{ForcedPlugin} in python; \texttt{SingleFrameAlgorithm}, \texttt{ForcedAlgorithm} in C++) and the measurement tasks that can be configured to run them (\texttt{SingleFrameMeasurementTask}, \texttt{ForcedMeasurementTask}, \texttt{CatalogCalculationTask}), as well as some concrete implementations of plugins (\texttt{ApertureFluxAlgorithm}, \texttt{BlendednessAlgorithm}, \texttt{CircularApertureFluxAlgorithm}, \texttt{GaussianFluxAlgorithm}, \texttt{LocalBackgroundAlgorithm}, \texttt{PeakLikelihoodFluxAlgorithm}, \texttt{PixelFlagsAlgorithm}, \texttt{PsfFluxAlgorithm}, \texttt{ScaledApertureFluxAlgorithm}, \texttt{SdssCentroidAlgorithm}, \texttt{SdssShapeAlgorithm}).
Each plugin has an associated config class, suffixed with \texttt{Config} in python or \texttt{Control} in C++ (e.g. \texttt{SdssCentroidAlgorithm} has \texttt{SdssCentroidControl}), used to configure parameters of that specific algorithm.

\subsubsection{Measurement plugins}
\label{sec:plugins}

Plugins are added to a \textit{registry}, so that they and their outputs can be referred to by a shorter common name that identifies the package it was defined in, for example \texttt{lsst.meas.base.SdssCentroidAlgorithm} is registered as \texttt{base\_SdssCentroid}.
This way, measurements produced by each plugin will have consistent, distinct names in the output schema, e.g. \texttt{base\_SdssCentroid\_x}, \texttt{base\_SdssCentroid\_y}, \texttt{base\_SdssCentroid\_flag}.

Measurement plugins often depend on each other, and must be run in a particular order.
This order is defined by the \texttt{executionOrder} config parameter, with smaller execution numbers being run first.
\texttt{BasePlugin} defines a list of named constants for particular cases, e.g. \texttt{FLUX\_ORDER} for plugins that require both a shape and centroid to have been measured.
Measurement plugins output their results to a \texttt{SourceCatalog} (TODO: crosslink to afw section!), which has a \textit{slot} system for predefined aliases to allow a plugin to get a value without knowing exactly what plugin originally computed that value, e.g. \texttt{slot\_Centroid} could point to \texttt{base\_SdssCentroid}, or some other plugin that measures centroids.

\subsubsection{SingleFrameMeasurementTask}
\label{sec:SingleFrameMeasurementTask}

Single frame measurement requires a catalog of detected source \texttt{Footprints}, which could still be blended, or could have been deblended (TODO: crosslink?).
When initialized, the task creates a schema from the configured plugins, which defines the contents of the output catalog and cannot be modified after initialization.

Before performing any measurement, this task replaces all sources with noise (via the \texttt{NoiseReplacer}) in the regions defined by their detected \texttt{Footprints}.
The task then loops over all "parent" sources (those that were not deblended and those that represent the un-deblended state of blends), and then loops over all "children" of parents (if any).
For each such source, the source footprint is re-inserted into the image, all measurement plugins are run, and the footprint is then replaced with noise again.
Then, for blended sources, the parent is inserted, measured (running plugins on both the parent and jointly on all the children via \texttt{measureN}), and again removed.

\subsubsection{ForcedMeasurementTask}
\label{sec:ForcedMeasurementTask}

Forced measurement uses the known pixel position of objects from a reference catalog to constrain measurements on another image.
Typically only photometric measurements are scientifically useful, as the centroid and shape are defined by the reference catalog, and transformed to the coordinate system of the image being measured on (e.g. shifting to the appropriate x/y origin, or transforming through the respective WCSs).
Other than this coordinate transformation, forced measurement proceeds much like single frame measurement above.
Two concrete implementations of the task include \texttt{ForcedPhotCcdTask} for single-visit images and \texttt{ForcedPhotCoaddTask} for coadd patch images, both using the output of a previous single frame measurement run on coadds as the reference catalog.

\subsection{meas\_algorithms}

\subsection{meas\_deblender}

\subsection{meas\_extensions\_convolved}

\subsection{meas\_extensions\_gaap}
\label{sec:meas_extensions_gaap}

\texttt{meas\_extensions\_gaap} implements the Gaussian Aperture and PSF photometry (GAaP) algorithm \citep{2008A&A...482.1053K}.
It is an aperture photometry algorithm designed to obtain consistent colors of extended objects (i.e., galaxies).
This is done by weighting each (pre-seeing) region of a galaxy by the same pre-defined 2D Gaussian function in all the bands and is thus largely insensitive to the seeing conditions in the different bands.
In practice, this is done by first convolving each object by a kernel (using the same tools described in \S\ref{sec:dia}) so that the PSF is Gaussian and is larger by about 15\% (this is configurable).
As a second step, each Gaussianized object is then weighted with a Gaussian aperture so that the effective pre-seeing Gaussian aperture is the same for all objects in all the bands.
The plugin is configured to use a series of circular Gaussian apertures, an elliptical Gaussian aperture (optionally) that matches the shape of the object in the reference band.

Although the two-step approach is motivated by the original implementation in \cite{2008A&A...482.1053K}, the implementation of this algorithm within the broader context of the measurement framework makes it different from the implementation used in the Kilo-Degree Survey \citep[KiDS;][]{2025arXiv250319439W}.
In particular, because neighboring objects are replaced with noise before measurement, Gaussianization of the PSF does not result in increased blending as mentioned in Appendix A2 of \cite{2015MNRAS.454.3500K}.
Furthermore, the uncertainty handling is different.
Correlations in noise introduced due to PSF-Gaussianization is included in the uncertainty estimates.
However, because only per-pixel noise variance is tracked, the noise treatment is forced to assume that the noise is uncorrelated to begin with which is not true on the coadds.
See \cite{DMTN-190} for more details on the implementation details.

Note that this measurements from this plugin do not produce total fluxes, but should only be used to obtain colors.
For total fluxes, measurements from \texttt{cModel} or \texttt{MultiProFit} (c.f., \S\ref{sec:meas_extensions_multiprofit}) are recommended.

\subsection{meas\_extensions\_photometryKron}

\subsection{PSF Modeling}

Within the pipeline, three distinct PSF models are defined:  \texttt{pcaPsf}, \texttt{PSFex}, and \texttt{Piff}.
Only \texttt{PSFex} and \texttt{Piff} are currently used.
\texttt{PSFex} is a fast, and less accurate PSF estimation and is wrapped within \texttt{meas\_extensions\_psfex}.
In contrast, \texttt{Piff} is a slightly slower, but more accurate PSF estimation that is incorporated in \texttt{meas\_extensions\_piff}. Both \texttt{meas\_extensions\_psfex} and \texttt{meas\_extensions\_piff} are described below.


\subsubsection{meas\_extensions\_psfex}

\subsubsection{meas\_extensions\_piff}


The \texttt{meas\_extensions\_piff} package is a wrapper around the PSF package  \texttt{Piff} used to estimate 
and compute the PSF (\citealt{2021ascl.soft02024J} and \citealt{2021MNRAS.501.1282J}) .  \texttt{Piff} is a 
modular package that supports various PSF models, interpolation schemes, coordinate systems, and can operate on 
a per-CCD basis or over the full field of view, as indicated by its name. The implementation 
within  \texttt{meas\_extensions\_piff} does not exploit the full modularity of \texttt{Piff}; instead, it closely follows the 
method used for cosmic shear analysis like in DES (\citealt{2021MNRAS.501.1282J} and \citealt{2025OJAp....8E..26S}).

The PSF model utilized is a  \texttt{PixelGrid}, and the interpolation is performed using  \texttt{BasisPolynomial} interpolation \citep{2021MNRAS.501.1282J}. 
Modeling is executed per CCD and can employ either pixel or sky coordinates. A key difference from  \texttt{PSFex} is that  \texttt{Piff} 
implements outlier rejection based on chi-squared criteria (see \citealt{2021MNRAS.501.1282J}  for more details).

Most of the configuration described here is adjustable through the  \texttt{PiffPsfDeterminerConfig} that are exposing some 
of the configurable parameters of \texttt{Piff} and can be fine-tuned for a 
dedicated survey. However, some important features that were implemented by \cite{2021MNRAS.501.1282J}  and 
\cite{2025OJAp....8E..26S} have not yet been enabled but will be available in the near future. While \cite{2021MNRAS.501.1282J} 
operates in sky coordinates with a WCS that includes CCD distortions such as treerings,  \texttt{meas\_extensions\_piff} can 
work in sky coordinates and incorporate WCS; as written, it does not, however, account for CCD distortions like treerings. Additionally, 
although \citealt{2025OJAp....8E..26S} incorporated a color correction to account for chromatic effects on the PSF,
 this correction has not yet been implemented in  \texttt{meas\_extensions\_piff}.

\subsection{meas\_extensions\_scarlet}

\subsection{meas\_extensions\_shapeHSM}
\label{sec:meas_extensions_shapeHSM}

The \texttt{meas\_extensions\_shapeHSM} package contains the plugins to measure the shapes of objects.
The plugins measure the moments of the sources and PSFs with adaptive Gaussian weights.
The algorithm was initially described in \citet{2003MNRAS.343..459H} and was modified later in \citet{2005MNRAS.361.1287M}.
The implementation of these algorithms lives within the \texttt{hsm} module of the GalSim package \citep{2015A&C....10..121R}.
\texttt{meas\_extensions\_shapeHSM} now interacts directly with the Python layer of GalSim to make the measurements.

The base plugin for measuring moments is the \texttt{HsmMomentsPlugin} and is the parent class of the \texttt{HsmSourceMomentsPlugin} and \texttt{HsmPsfMomentsPlugin} which are specialized to measure on the sources (and objects) and PSFs respectively.
\texttt{HsmSourceMomentsRoundPlugin} is a further specialized plugin that measures the moments with circular Gaussian weights instead of the elliptical ones in \texttt{HsmSourceMomentsPlugin}.
The \texttt{HsmPsfMomentsDebiasedPlugin} adds noise to the PSF image to degrade it to have the same signal-to-noise ratio (SNR) as the source image.
This makes the ellipticity calculated from this plugin have the same bias as the source ellipticity
The PSF moments from this plugin should be used when calculating ellipticity residuals so the bias is largely cancelled.
Having the various specializations as distinct plugins allows an object to be measured under different configurations simultaneously and included in the output catalogs.

In addition to the plugins that measure (adaptive) weighted moments, there are also a series of \texttt{HsmShape} plugins to estimate the PSF-corrected ellipticities of objects.
In particular, the outputs from \texttt{HsmShapeRegaussPlugin} have been used to measure weak gravitation lensing signals in the Hyper Suprime-Cam SSP data \citep{2018PASJ...70S..25M, 2022PASJ...74..421L}.

\subsection{meas\_extensions\_simpleShape}

\subsection{meas\_extensions\_trailedSources}

\subsection{meas\_modelfit}

\subsection{meas\_extensions\_multiprofit}

\subsection{Reliability Scoring}
\label{sec:reliability}

The \texttt{meas\_transiNet} package determines a numerical score for input cutout images using pre-trained machine-learning models.
Image differencing may produce false detections, so time-domain surveys chacteristically use machine learning classifiers to distinguish astrophysical sources from artifacts \citep[``Real/Bogus;'' e.g.,][]{2012PASP..124.1175B, 2015AJ....150...82G, 2019MNRAS.489.3582D}.

The \texttt{meas\_transiNet} defines ``model packages'' that consist of a python architecture class, a PyTorch \citep{NEURIPS2019_bdbca288} weights file, and associated metadata.
The inference task may be configured to load a model package from disk or from the Butler.

The \texttt{RBTransiNetTask} PipelineTask takes as input three square cutouts of configurable size from the science, template, and difference images centered on the location of a source.
These images are concatenated, batched into Torch blobs, and passed to the model for inference.
Either CPU or GPU backends may be used for inference.
The output of the task is a single float ranging from 0--1 for each cutout triplet, with higher values indicating that the DIASource is more likely to be astrophysical.
These reliability scores are then joined with the DIASource catalogs by a later transformation task.
Detailed discussion of the model architecture, training, and performance will be presented in T.\ Acero Cuellar et.\ al (in prep.).

