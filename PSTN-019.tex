\documentclass[twocolumn]{aastex631}


\shorttitle{The LSST Science Pipelines}
\shortauthors{Rubin Observatory Data Management Pipeline Developers}

\begin{document}

\title{The LSST Science Pipelines Software: Optical Survey Pipelined Reduction and Analysis Environment}

\AuthorCallLimit=999
\input{authors}
\collaboration{100}{The Rubin Observatory Science Pipelines Team}


\begin{abstract}

The NSF-DOE Vera C.\ Rubin Observatory will execute the Legacy Survey of Space and Time (LSST) as its prime mission,  producing a series of data releases over the ten-year survey.
The LSST Science Pipelines Software, the Optical Survey Pipeline Reduction and Analysis Environment (OSPRAE), will be used to create these data releases and to perform the nightly prompt processing and alert production.
This paper provides an overview of the LSST Science Pipelines Software, describing the components and their integration into pipelines that generate science-ready data products.

\end{abstract}


\keywords{%
    Astrophysics - Instrumentation and Methods for Astrophysics
    ---
    methods: data analysis
    ---
    methods: miscellaneous
}

\section{Introduction}
\label{sec:intro}

The NSF-DOE Vera C.\ Rubin Observatory will be performing the 10-year Legacy Survey of Space and Time \citep[LSST;][]{2019ApJ...873..111I} starting in 2026.
Rubin Observatory is located on Cerro Pach\'on in Chile and consists of the 8.4\,m Simonyi Survey Telescope \citep{2022SPIE12182E..0WT} with the 3.2-gigapixel LSSTCam survey camera \citep{10.71929/rubin/2571927,2024SPIE13096E..1SR} performing the main survey and the Rubin Auxiliary Telescope \citep{2020SPIE11452E..0UI} providing supplementary atmospheric calibration data using the LATISS instrument \citep{10.71929/rubin/2571930}.
The Data Management System \citep[DMS;][]{2025ASPC..538...53O} is designed to handle the flow of data from the telescope, approaching 10\,TB per night.
It produces the prompt data products, including alerts, and prepares annual data releases.
A central component of the DMS is the LSST Science Pipelines software, which provides the algorithms and frameworks required to process LSST data  and produce science-ready data products, including coadds, difference images, and catalogs.

The LSST Science Pipelines software consists of the building blocks and infrastructure required to construct high performance pipelines to process the data from the LSST.
It use used both for Data Release Production (DRP) and Alert Production (AP).
It has been under development since at least 2004 \citep{2004AAS...20510811A} and has evolved significantly over the years as the project transitioned from prototyping \citep{2010SPIE.7740E..15A} and entered into formal construction \citep{2017ASPC..512..279J}.
The software is open-source and is designed to be usable by other optical telescopes, a capability that has been demonstrated with the processing of data from Hyper Suprime-Cam (HSC) on the Subaru Telescope in Hawaii \citep{2022PASJ...74..247A} as well as data from additional instruments used for internal testing.
There have also been plugins developed by the community for other instruments.
It was used to make the first public data release from Rubin Observatory, the LSST Data Preview 1 \citep{RTN-095,10.71929/rubin/2570308}, which consisted of data from the LSST Commissioning Camera \citep[LSSTComCam;][]{10.71929/rubin/2561361,2022SPIE12184E..0JS}.

In this paper we provide an overview of the software system, dividing it into four rough tiers:
\begin{itemize}
\item infrastructure consisting of low-level utility code \secrefp{sec:support}, data access, configuration, and execution frameworks \secrefp{sec:middleware}, and the core algorithmic primitives \secrefp{sec:core};
\item reusable mid-level algorithmic components \secrefp{sec:components};
\item high-level tasks and pipelines, along with some details of the algorithms specific to them \secrefp{sec:tasks-and-pipelines};
\item analysis and validation tooling \secrefp{sec:analysis}.
\end{itemize}
There are no sharp boundaries between these tiers; they are better considered to be a loose organizational aid than any kind of formal classification.
We do not include details of the science validation of the individual algorithms, and do not attempt to cover components of the system in uniform detail; instead we focus attention on those that are unique or novel or not well described elsewhere.
The other components of the LSST DMS, such as the workflow system \citep{2025ASPC..538..325G,2024EPJWC.29504026K}, the Qserv database \citep{Wang:2011:QDS:2063348.2063364,2025ASPC..538..114M} and the Rubin Science Platform \citep{LSE-319,2024ASPC..535..227O}, are not covered in this paper.

\section{Fundamentals}

The LSST Science Pipelines software is written in Python with C++ used for high performance algorithms and for core classes that are usable in both languages.
We recently dropped Python 2 and adopted Python 3 \citep{2017arXiv171200461J}, requiring a minimum version of Python 3.6.
The C++ layer can use C++14 features and we use pybind11 to provide the interface from Python to C++.
Additionally, the C++ layer uses \texttt{ndarray} to allow seamless passing of C++ arrays to and from Python \texttt{numpy} arrays.
This compatibility with \texttt{numpy} is important in that it makes LSST data structures available to standard Python libraries such as Scipy and Astropy \citep{2016SPIE.9913E..0GJ,2018arXiv180102634T}.

Although all the software uses the \texttt{lsst} namespace, the code base is split into individual Python products in the LSST GitHub organization\footnote{\url{https://github.com/lsst}} that can be installed independently and which declare their own dependencies.
These dependencies are managed using the EUPS system \citep{EUPS,2018SPIE10707-10J}.
Each product is built using the SCons system \citep{2005Scons1377085} with LSST-specific extensions provided in the \texttt{sconsUtils} package enforcing standard build rules.

For logging we use Log4CXX wrapped in the \texttt{lsst.log} package to make it look more like standard Python logging whilst also supporting deferred string formatting such that log messages are only formed if the log message level is sufficient for the message to be logged.
Finally, we also provide some LSST-specific exceptions that can be thrown from C++ code and caught in Python.

As of April 2018, the Science Pipelines software is approximately 290,000 lines of Python and 225,000 lines of C++.\footnote{Line counts include comments but not blank lines. Python interfaces are implemented using \texttt{pybind11} and that is counted as C++ code. For the purposes of this count Science pipelines software is defined as the \texttt{lsst\_distrib} metapackage and does not include code from third party packages.}

\subsection{Unit Testing and Code Coverage}

Unit testing and code coverage are critical components of code quality \citep{2018SPIE10707-10J}.
Every package comes with unit tests written using the standard \texttt{unittest} module.
We run the tests using \texttt{pytest} \citep{pytest} and this comes with many advantages in that all the tests run in the same process and requiring global parameters to be well understood, test can be run in parallel in multiple processes, plugins can be enabled to extend testing, and a test report can be created giving details of run times and test failures.
Coding standards compliance with PEP\,8 \citep{pep8} and code coverage are enabled using \texttt{pytest} plugins, and we also check for leaked file descriptors during tests.

\section{Data Access Abstraction}

\subsection{Butler}

\subsection{Instrument Abstractions: Obs Packages}

\section{Core Infrastructure Libraries}

\subsection{Region Handling}

\texttt{geom} and \texttt{sphgeom}?

Use ICRS coordinates everywhere.
All coordinate transformations are done within Astropy.

\subsection{Time and Hierarchical Data Structures}

\texttt{daf\_base}.

Use Datetime only to store times in C++ objects.
Use \texttt{astropy.time} for all other time handling, following the recommendations from \citet{2016SPIE.9913E..0GJ}.

\texttt{PropertySet} and \texttt{PropertyList} to allow \texttt{dict}-like data structures to be passed from Python to C++ and back again.

\subsection{Astronomy Framework}
\label{sec:afw}
\texttt{afw}

AST library \citep{2016A&C....15...33B} backs the world coordinate system handling.

\texttt{coadd\_utils} ?

\section{Instrument Signature Removal}
\label{sec:isr}

\section{Measurement System}
\label{sec:meas}

Measurement plugin system.

\texttt{meas\_base} and \texttt{meas\_algorithms}


\subsection{meas\_deblender}
\subsection{meas\_extensions\_convolved}
\subsection{meas\_extensions\_gaap}
\subsection{meas\_extensions\_photometryKron}
\subsection{meas\_extensions\_piff}
\subsection{meas\_extensions\_psfex}
\subsection{meas\_extensions\_scarlet}
\subsection{meas\_extensions\_shapeHSM}
\label{sec:meas_extensions_shapeHSM}
\texttt{meas\_extensions\_shapeHSM} package contains the plugins to measure the shapes of objects.
The plugins measure the moments of the sources and PSFs with adaptive Gaussian weights.
The algorithm was initially described in \cite{2003MNRAS.343..459H} and was modified later in \cite{2005MNRAS.361.1287M}.
The implementation of these algorithms lives within the \texttt{hsm} module of the GalSim package \citep{2015A&C....10..121R}.
\texttt{meas\_extensions\_shapeHSM} now interacts directly with the Python layer of GalSim to make the measurements.

The base plugin for measuring moments is the \texttt{HsmMomentsPlugin} and is the parent class of the \texttt{HsmSourceMomentsPlugin} and \texttt{HsmPsfMomentsPlugin} which are specialized to measure on the sources (and objects) and PSFs respectively.
\texttt{HsmSourceMomentsRoundPlugin} is a further specialized plugin that measures the moments with circular Gaussian weights instead of the elliptical ones in \texttt{HsmSourceMomentsPlugin}.
The \texttt{HsmPsfMomentsDebiasedPlugin} adds noise to the PSF image to degrade it to have the same signal-to-noise ratio (SNR) as the source image.
This makes the ellipticity calculated from this plugin have the same bias as the source ellipticity
The PSF moments from this plugin should be used when calculating ellipticity residuals so the bias is largely cancelled.
Having the various specializations as distinct plugins allows an object to be measured under different configurations simultaneously and included in the output catalogs.

In addition to the plugins that measure (adaptive) weighted moments, there are also a series of \texttt{HsmShape} plugins to estimate the PSF-corrected ellipticities of objects.
In particular, the outputs from \texttt{HsmShapeRegaussPlugin} have been used to measure weak gravitation lensing signals in the Hyper Suprime-Cam SSP data \citep{2018PASJ...70S..25M, 2022PASJ...74..421L}.
\subsection{meas\_extensions\_simpleShape}
\subsection{meas\_extensions\_trailedSources}
\subsection{meas\_modelfit}
\subsection{meas\_transiNet}

\section{Difference Imaging}
\label{sec:diffim}

Difference imaging is implemented in \texttt{ip\_diffim}, and is divided into three steps.
First, a base template image is constructed with \texttt{getTemplate} by warping previously-generated coadded images to the WCS and bounding box of the science image.
Then the warped template is subtracted from the science image using one of several available algorithms in \texttt{subtractImages}, which produces a temporary difference image.
Finally, peaks are detected on the difference image and DiaSources are measured in \texttt{detectAndMeasure}.
The final difference image with updated mask planes is written along with the DiaSource catalog.

\subsection{subtractImages}
The primary implementation of image subtraction used by \texttt{subtractImages} is based on \cite{1998ApJ...503..325A}, and uses spatially-varying Gaussian basis functions for the fit.
The PSF-matching kernel can be constructed for either the science or the template image, and the resulting difference image is decorrelated \citet{DMTN-021}.
Optionally, the science image can be preconvolved with its own PSF before PSF-matching, producing a Score image analogous to \cite{Zackay_2016}.

\subsection{detectAndMeasure}
Positive and negative peaks are detected by thresholding the Score image if it is available.
Otherwise, the difference image is smoothed with a Gaussian of the same width as the PSF of the science image, and thresholds are taken on the smoothed image.
Contiguous pixels around each peak that are statistically brighter than the background are grouped into source footprints, and any overlapping footprints are merged.
Footprints that contain both a positive and a negative peak are fit as dipoles.
The dipole fit simultaneously solves for the negative and positive lobe centroids and fluxes using non-linear least squares minimization.
DiaSources that are not classified as dipoles instead fall back on an SDSS-style centroid (\cite{Pier_2003}).
Finally, all configured measurement plugins are run, including HSM shape measurements (\cite{2003MNRAS.343..459H} and \cite{2005MNRAS.361.1287M} ) and trailed source measurements.
\section{Astrometric and Photometric Calibration}

\section{Pipeline Support}

\subsection{Tasks and SuperTask}

\subsection{Catalog Schemas}
\label{sec:schemas}

Pipeline products must be transformed from the internal data model to the public data model defined in \citet{LSE-163}.
A set of YAML files in the \texttt{pipe\_tasks}\footnote{\url{https://github.com/lsst/pipe\_tasks}} repository are used for transforming the internal pipelines representation of the data to a standardized parquet output format.
These parquet files are continually validated against an appropriate schema to ensure that the column names and types are correct as the pipelines codebase evolves, through repositories such as \texttt{ci\_imsim}\footnote{\url{https://github.com/lsst/ci\_imsim}} and \texttt{ci\_hsc}\footnote{\url{https://github.com/lsst/ci\_hsc}}.
For data previews and releases, the parquet files are then ingested into the Qserv database, where the data catalogs are stored.

The public data model is defined by a set of files in the Felis \citep{2024arXiv241209721M} YAML format which describe the schema of a data catalog, including its tables, columns, constraints and metadata.
These are collectively referred to as the Science Data Model (SDM) schemas.
The YAML files are managed with the \texttt{sdm\_schemas} repository\footnote{\url{https://github.com/lsst/sdm\_schemas}} with all changes validated by GitHub workflows.
The schemas corresponding to Science Pipelines output are continually evolving with the pipelines codebase, so, for instance, column names and types may be updated to reflect changes to the internal data model.
Schemas for data previews and releases represent a snapshot of the public data model at the time of the release and would typically only be updated with bug fixes, minor changes, or updates and additions to the metadata.
For public-facing data catalogs, the Felis representation is used to generate a TAP\_SCHEMA database describing the tables and columns available in the TAP service \citep{2019ivoa.spec.0927D} and serves as a source of documentation.
User data access to the data catalogs is provided primarily through the Astronomical Data Query Language (ADQL) through the TAP service, with the tables and columns being validated against the TAP\_SCHEMA as part of the query execution process.

\subsection{Display Abstractions}
\label{sec:display}

The Python object representing an image with metadata is a bespoke object not understand by generic tooling.
To display an image we provide a display abstraction layer that allows the image to be displayed and graphics overlaid by using a plugin mechanism.

In some plugins the pixel data can be extracted from the exposure object and sent directly to display, in other plugins we form a simple single HDU FITS image (possibly with simplified world coordinates) and pass the temporary FITS file to the display system.

There a currently plugins for matplotlib \citep{2007CSE.....9...90H}, Firefly \citep{2020ASPC..527..243R}, SAOImage DS9 \citep{2003ASPC..295..489J}, and Ginga \citep[][via Astrowidgets]{2013ASPC..475..319J}.

\section{Data Analysis}
\label{sec:analysis}

\texttt{analysis\_tools}

\texttt{verify}

\texttt{faro} --- \textbf{do not document this as we are no longer using it for primary metrics calculation}.

\subsection{Validating the Science Pipelines}
\label{sec:validation}

We use small, of order of a few gigabyte, datasets that can be processed as part of continuous integration.
These take of order an hour to process.
There are regular re-processings of standard datasets that can take a few days to process.
For formal data releases there are additional metrics calculated and a test report is issued, such as the one made available with release 28.0 \citep{DMTR-451}.

\subsection{Source Injection}
\label{sec:source_injection}

The \texttt{source\_injection} package contains tools designed to assist in the injection of synthetic sources into scientific imaging.
Source injection is a powerful tool for testing the algorithmic performance of the LSST Science Pipelines, generating measurements on synthetic sources where the truth is known and facilitating subsequent quality assurance checks.
Synthetic source generation and injection capability is provided by the \textsc{GalSim} software package \citep{2015A&C....10..121R}.
An example showcasing the injection of a series of synthetic Sérsic sources into an HSC i-band image is shown in Figure \ref{fig:source_injection_example}.

\begin{figure}
    \centering
    \includegraphics[width=\linewidth]{figures/t9813p42i_zoom_sersic_pre_injection}
    \includegraphics[width=\linewidth]{figures/t9813p42i_zoom_sersic_post_injection}
    \caption{
        An HSC i-band cutout from tract 9813, patch 42, showing before (top) and after (bottom) the injection of a series of synthetic Sérsic sources.
        Images are ~100 arcseconds on the short axis, log scaled across the central 99.5\% flux range, and smoothed with a Gaussian kernel of FWHM 3 pixels.
    }
    \label{fig:source_injection_example}
\end{figure}

Synthetic sources can be injected into any imaging data product output by the LSST Science Pipelines, including visit-level exposure-type or visit-type datasets (i.e., datasets with the dimension \texttt{exposure} or \texttt{visit}), or into a coadd-level coadded dataset.
These injection tasks are defined in \texttt{ExposureInjectTask}, \texttt{VisitInjectTask} and \texttt{CoaddInjectTask}, respectively.
Each task operates similarly: read in an injection catalog containing the parameters of the sources to be injected, generate sources using \textsc{GalSim}, and inject them into the input image.
An additonal mask plane (\texttt{INJECTED} by default) is appended to the image mask to identify pixels which have been touched by injected sources.
Optional modifications to the noise profiles of injected sources and the variance plane of the image can also be performed.

With \textsc{GalSim} we have the capacity to generate synthetic sources of varying profile types, including Gaussian, exponential and Sérsic profiles \citep{1963BAAA....6...41S, 1968adga.book.....S}, each convolved with the local PSF.
We also have the option to inject scaled versions of the PSF model itself in order to simulate stars.
If preferred, a pre-generated FITS image of a source can be injected instead of a model generated by \textsc{GalSim}, allowing for the injection of complex sources or postage stamp cutouts of real data.

Alongside the primary injection tasks, a suite of helper tools are also provided to optionally assist in the generation of synthetic source catalogs and injection pipelines.
Fully qualified source injection pipeline definition YAML files are normally constructed using an existing pipeline as a baseline reference.
A user specifies which dataset type they would like to inject synthetic sources into, and the \texttt{source\_injection} package generates a new pipeline definition YAML file that includes the correctly configured source injection task.
By default, all tasks in the pipeline downstream of the point at which source injection occurs are modified such that their connection names are prefixed with \texttt{injected\_}.
This ensures that an injected dataset is not confused with the original dataset when stored together in a common collection.

Once source injection has completed, the source injection task will output two dataset types: an injected image, and an associated injected catalog.
The injected image is a copy of the original image with the injected sources added.
The injected catalog is a catalog of the injected sources, with the same schema as the original catalog and additional columns describing per-source source injection success outcomes.


\section{Conclusions}
\label{sec:conclusions}

The LSST Science Pipelines Software has been developed over 20 years to support the processing of the Legacy Survey of Space and Time.
It has been used to process formal data releases from both Hyper Suprime-Cam and the Rubin Observatory's LSSTComCam and is now being used to process LSSTCam commissioning data.
The software is designed to be extensible and reusable, supporting a plugin architecture that allows new algorithms to be added without modifying the core codebase and includes a dataset tracking system and graph builder that supports scaling of processing on large batch systems.
The success of this architecture is further demonstrated by its adoption by several other astronomical surveys, including VISTA's VIR-CAM, WFST, GOTO, and SPHEREx, for processing their data.
The software will continue to be developed and evolve as new data releases are made and to support the community.


\begin{acknowledgments}
This material is based upon work supported in part by the National Science Foundation through Cooperative Agreement AST-1258333 and Cooperative Support Agreement AST-1202910 managed by the Association of Universities for Research in Astronomy (AURA), and the Department of Energy under Contract No. DE-AC02-76SF00515 with the SLAC National Accelerator Laboratory managed by Stanford University.
Additional Rubin Observatory funding comes from private donations, grants to universities, and in-kind support from LSSTC Institutional Members.
\end{acknowledgments}

\facilities{Rubin (LSSTCam), RubinAux (LATISS)}
\software{%
ndarray (\url{https://github.com/ndarray/ndarray}),
astropy \citep{2022ApJ...935..167A},
pytest \citep{pytest},
matplotlib \citep{2007CSE.....9...90H},
galsim \citep{2015A&C....10..121R},
numpy \citep{Harris2020},
gbdes \citep{2022ascl.soft10011B},
Starlink's \citep{2022ASPC..532..559B} AST \citep{2016A&C....15...33B},
fgcm (\url{https://github.com/erykoff/fgcm}),
}

\bibliography{local,lsst,lsst-dm,refs_ads,refs,books}

\end{document}
