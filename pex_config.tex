\subsection{Pex Config}

Pex Config is the foundational configuration system for the LSST Rubin Observatory’s ambitious science pipelines.
It’s far more than a simple parameter parser; it’s a framework that mediates between diverse configuration sources and the complex software that processes astronomical data.
At its core, Pex Config functions as an intermediate representation, decoupling the pipelines from the specifics of configuration file formats (like YAML, JSON) and providing a unified, Python-native interface to all configurable parameters.
This abstraction is critical for maintainability, allowing the underlying file formats and or execution systems to evolve without impacting the pipeline code.
It also provides a mechanism to deprecate configurables which will change in future versions of the software stack, allowing users an easy migration path.


The design of Pex Config centers around the concepts of “Fields” and “Config” objects.
Fields represent individual configurable values – things like exposure times, image quality thresholds, or database connection strings.
Each Field is strongly typed, supporting a variety of data types (integers, floats, strings, booleans, lists, etc.).
Config objects, on the other hand, are containers that group related Fields together, creating logical units of configuration.
One of the highlights of Pex Config is its composability.
Config objects can be nested within other Config objects using a special “ConfigField,” allowing for the creation of complex, hierarchical configuration trees that mirror the structure of the pipelines themselves.
This allows for modularity and reuse of configuration components across different parts of the system.


A strength of Pex Config is its flexible application of configuration values.
Values can be set at multiple stages: via command-line arguments, loaded from configuration files, or defined directly within the pipeline code.
Importantly, these stages are applied progressively, with later stages overriding earlier ones.
This allows for a powerful combination of default settings, user-defined customizations, and dynamic adjustments.
Mechanisms also exist to apply values to all instances of a particular Config object within a tree, simplifying the management of shared parameters and ensuring consistency.


Beyond runtime configuration, Pex Config is deeply concerned with data provenance and reproducibility.
It provides mechanisms for persisting and restoring configuration values, allowing for complete tracking of pipeline parameters used in a particular data processing run.
Crucially, it also maintains a history of each Field’s value, recording when and where it was set – whether via the command line, a configuration file, or programmatically.
This detailed history is invaluable for debugging, auditing, and ensuring the reproducibility of scientific results.
The system also incorporates robust validation mechanisms, enabling checks on individual Fields and groups of values before they are used by the pipelines, preventing errors and ensuring data quality.
Validation can range from simple type checking, ensuring values fall within acceptable ranges or specific patters, to complex custom functions that enforce specific constraints.


Finally, Pex Config is designed with documentation in mind.
All Fields and Config objects can be richly documented using docstrings and documentation attributes.
This documentation structure is not only readable by humans but can also be parsed by automated tools to generate comprehensive documentation pages, eliminating the need for manual documentation creation.
This ensures that the configuration system is well-documented and easy to understand, even for new developers.
