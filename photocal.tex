\subsection{Photometric Calibration}
\label{sec:Photometric Calibration}

\subsubsection{PhotoCal}
\label{sec:PhotoCal}

TODO: Eli should look at this?
Single frame photometric calibration is performed by \texttt{PhotoCalTask}, in the \texttt{pipe\_tasks} package.
This task requires that the input catalog come from an image with a good astrometric solution.
The catalog to be calibrated is down-selected to be bright ($S/N>10$), well measured, PSF-like sources which are then matched to a reference catalog.
The matched sources have their instrumental fluxes converted into rough magnitudes, which are are iteratively compared with the reference catalog magnitudes using a sigma-clipping algorithm, to fit a single magnitude zero point to the whole image.

Because \texttt{PhotoCalTask} only requires a single image, it is suitable for use during Alert Production, which does not have access to the entire focal plane.
This single frame photometric fit is sufficient for initial calibration and difference imaging (assuming the flat field calibrations applied during ISR are a close match to the true instrument response), but during DRP we perform a full focal plane fit with \texttt{gbdes}.

\subsection{fgcmcal}
\label{sec:fgcmcal}

Global photometric calibration is computed by use of the Forward Global Calibration Method~\citep[FGCM][]{2018AJ....155...41B}, as adopted for LSST Science Pipelines~\citep{SITCOMTN-086}.
This global calibration algorithm makes use of repeated observations of stars in all $ugrizy$ bands, combining a forward model of the atmospheric parameters with instrumental throughputs measured with auxiliary information.
In this way we simulateously constrain the atmospheric model as well as standardized top-of-atmosphere (TOA) star fluxes over a wide range of star colors, including full chromatic corrections from the instrument and atmosphere.

Running \texttt{fgcmcal} first requires generating a look-up table.
The input to the look-up table includes the effect of a MODTRAN~\citep{10.1117/12.366388} atmospheric model at the elevation of the observatory, as well as the throughput as a function of wavelength and position from the optics, filters, and detector quantum efficiency.
The quality of the output (in terms of repeatability of bright isolated stars across a wide range of colors) depends on the knowledge of the instrumental throughput.

The primary goal of fgcmcal is to provide a uniform relative photometric calibration of the survey.
For ``absolute'' (relative) calibration, a reference catalog can be used as an additional constraint on the fit.
Thus, the overall throughput output by \texttt{fgcmcal} depends on the reference catalog.
This can be checked with (e.g.) specific white dwarfs or CALSPEC~\citep{2007ASPC..364..315B} stars in the survey.
However, the relative spatial and chromatic calibration of the \texttt{fgcmcal} calibration means that the absolute calibration reduces to a set of 6 numbers (one for each band, or one overall throughput and 5 absolute colors).
