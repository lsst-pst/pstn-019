\section{Pipelines}
\label{sec:pipe}

\subsection{Pipeline Support}

Tasks and PipelineTask overview.

The \texttt{Task} Python class provides a standard interface for how to execute an algorithm.
The \texttt{PipelineTask} variant provides stronger guarantees on configuration and provides a means by which the pipeline execution framework can determine how to link a task into a pipeline and how to determine what type of data should be read from a Butler and what should be written out to a Butler.

Maybe describe \texttt{pex\_config} because it's not described anywhere.

\subsection{Task library}

\texttt{pipe\_tasks}
\texttt{drp\_tasks}

\subsection{Pipeline Collections}

\texttt{drp\_pipe}

The \texttt{ap\_pipe} package defines the pipeline(s) to be used for real-time Alert Production processing (\ref{}).
These pipelines include instrument signature removal (\S\ref{sec:isr}), calibration (\S\ref{}), measurement plugins (\S\ref{sec:meas}), image differencing (\S\ref{sec:diffim}), source association (\S\ref{sec:association}), and alert generation (\S\ref{sec:alerts}).
Some of these tasks are shared with the pipelines in \texttt{drp\_pipe}, but configured to prioritize speed over strict quality; for example, they use a minimal set of measurement plugins.

\texttt{ap\_pipe} currently has pipeline variants for LATISS, the Rubin Observatory simulators, Hyper-SuprimeCam, and the Dark Energy Camera.
Because these variants serve as testbeds for AP-specific algorithms and configuration settings, they are, as much as possible, the ``same'' pipeline, differing almost entirely in loading instrument defaults from \texttt{obs} packages (\S\ref{sec:obs_packages}).
The only other customization is an extra task for handling DECam's inter-chip crosstalk, which does not have an equivalent for Rubin instruments.
