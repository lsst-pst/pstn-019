\subsection{meas\_extensions\_gaap}
\label{sec:meas_extensions_gaap}

\texttt{meas\_extensions\_gaap} implements the Gaussian Aperture and PSF photometry (GAaP) algorithm \citep{2008A&A...482.1053K}.
It is an aperture photometry algorithm designed to obtain consistent colors of extended objects (i.e., galaxies).
This is done by weighting each (pre-seeing) region of a galaxy by the same pre-defined 2D Gaussian function in all the bands and is thus largely insensitive to the seeing conditions in the different bands.
In practice, this is done by first convolving each object by a kernel (using the same tools described in \S\ref{sec:dia}) so that the PSF is Gaussian and is larger by about 15\% (this is configurable).
As a second step, each Gaussianized object is then weighted with a Gaussian aperture so that the effective pre-seeing Gaussian aperture is the same for all objects in all the bands.
The plugin is configured to use a series of circular Gaussian apertures, an elliptical Gaussian aperture (optionally) that matches the shape of the object in the reference band.

Although the two-step approach is motivated by the original implementation in \cite{2008A&A...482.1053K}, the implementation of this algorithm within the broader context of the measurement framework makes it different from the implementation used in the Kilo-Degree Survey \citep[KiDS;][]{2025arXiv250319439W}.
In particular, because neighboring objects are replaced with noise before measurement, Gaussianization of the PSF does not result in increased blending as mentioned in Appendix A2 of \cite{2015MNRAS.454.3500K}.
Furthermore, the uncertainty handling is different.
Correlations in noise introduced due to PSF-Gaussianization is included in the uncertainty estimates.
However, because only per-pixel noise variance is tracked, the noise treatment is forced to assume that the noise is uncorrelated to begin with which is not true on the coadds.
See \cite{DMTN-190} for more details on the implementation details.

Note that this measurements from this plugin do not produce total fluxes, but should only be used to obtain colors.
For total fluxes, measurements from \texttt{cModel} or \texttt{MultiProFit} (c.f., \S\ref{sec:meas_extensions_multiprofit}) are recommended.
