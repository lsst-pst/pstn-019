\section{Core Algorithmic Primitives and Data Structures}
\label{sec:core}

The high-level algorithms in the LSST Science Pipelines are largely built from algorithmic primitives and data structures implemented in the \texttt{afw} package.
\texttt{afw} is a large, complex, C++-heavy suite of multiple libraries that sometimes suffer from historical idiosyncrasies, but are nevertheless extremely powerful and well optimized.
These include (but are not limited to):
\begin{itemize}
\item \texttt{afw.geom} holds our high-level geometry primitives.  This includes composable coordinate transforms and world coordinate systems (WCS).
    It also includes \texttt{SpanSet}, a run-length encoding (RLE) description of a set of pixels in a 2-d image, a simple \texttt{Polygon} class, and routines for working with various ellipse parameterizations.
\item \texttt{afw.math} includes convolution, resampling, general-purpose interpolation, statistics, and least-squares fitting algorithms.
\item \texttt{afw.detection} contains threshold-based detection on images and point-spread function (PSF) model interfaces.
    This includes the \texttt{Footprint} class, which combines a \texttt{SpanSet} with a list of peaks to represent either a single source detection or a group of blended sources.
\item \texttt{afw.image} centers around the \texttt{Exposure} class, which combines an image (\texttt{Image}), bitmask (\texttt{Mask}), and variance image with the many objects used to astrophysically characterize an observation or coadd (PSF, WCS, aperture corrections, etc).
\item \texttt{afw.table} holds data structures for tabular data, with both row- and column-based views.  The \texttt{afw.table.io} package defines a framework for persisting arbitrary objects to a series of FITS binary table HDUs, which is used in the on-disk form of the \texttt{Exposure} class.
\item \texttt{afw.cameraGeom} provides a hierarchical description of large-format photometric cameras (like LSSTCam, HSC, or DECam), including optical distortions, focal plane layouts, and amplifier regions.
\end{itemize}
Most of the algorithms and data structures in \texttt{afw} are implemented there directly, and often represent evolutions of concepts first developed in the SDSS \emph{Photo} pipeline or the Pan-STARRS Image Processing Pipelines.
Others delegate to third-party libraries, particularly Eigen (linear algebra), the Gnu Scientific Library (interpolation, random numbers), Boost (image iterators and geometry), CFITSIO (image and table I/O) and Starlink AST \citep[coordinate systems and transforms;][]{2016A&C....15...33B}.

Even lower-level data structures are defined in a handful of packages just below \texttt{afw}.
The \texttt{sphgeom} package is used for spherical geometry calculations, sky-based regions, and hierarchical sky pixelization schemes, while \texttt{geom} provides simple 2-d Euclidean \texttt{Point}, \texttt{Extent}, and \texttt{Box} types in both integer- and floating-point variants.
\texttt{geom} also includes linear transforms and (for historical reasons) its own angle-manipulation and sky coordinate type.

C++ mapping types (with Python bindings) and date/time objects are defined in \texttt{daf\_base}.
The \texttt{DateTime} package is used in our C++ data structures mostly to represent TAI times.
The \texttt{PropertySet} represents a hierarchical key/value data structure whereas \texttt{PropertyList} is a flat data structure that is used to represent a FITS header and supports multi-valued keys and key comments.

Another small set of core packages sits just above \texttt{afw}:
\begin{itemize}
\item \texttt{skymap} defines interfaces and a few implementations of our system for mapping the sky onto a set of slightly-overlapping image-friendly projections and tiles for coaddition.
    Each distinct projection in a skymap is called a \texttt{tract}, and each \texttt{tract} is further divided into multiple \texttt{patches}.
\item \texttt{shapelet} includes optimizated evaluation of Gauss-Hermite and Gauss-Laguerre functions and their derivatives.
\end{itemize}

A substantial fraction of our core packages predate the now-ubiquitous Astropy package, and in some cases we now prefer to use Astropy types in \emph{most} new code (date/time representations and tables in particular) and especially public interfaces, following the recommendations from \citet{2016SPIE.9913E..0GJ}.
Fully retiring core libraries that have Astropy counterparts is at best a long-term project, however, due to our continued need for these objects in considerable amounts of C++ that has no equivalent in Astropy (or anywhere else).
