\subsubsection{meas\_extensions\_psfex}\label{sec:meas_extensions_psfex}

The \texttt{meas\_extensions\_psfex} package provides an interface to a patched version of the PSFEx tool \citep{2011ASPC..442..435B} for modeling spatially varying point spread functions (PSFs).
It uses PSF candidates typically selected from detected sources using a configurable star selector that selects clean, isolated stars.
At its core is \texttt{PsfexPsfDeterminerTask}, which prepares these selected stars for input into PSFEx, runs the external binary, and converts the output into an LSST-specific PSF object (\texttt{PsfexPsf}).
Key parameters such as spatial interpolation order and oversampling ratio are controlled via \texttt{PsfexPsfDeterminerConfig}.

For each CCD in the focal plane, PSFEx independently models the PSF as a linear combination of basis vectors and captures spatial variation using polynomial interpolation.
\texttt{PsfexStarSelectorTask} offers a built-in mechanism for star selection using strict cuts on signal-to-noise ratio, FWHM range, ellipticity, and quality flags.
The package raises specific exceptions for common failure modes: when no stars are available, when none pass quality cuts, or when too few good stars remain to support the required model complexity--as determined by the degrees of freedom needed for the fit.
In the latter case, the insufficient sample size forces the PSFEx model to reduce its polynomial degree to an unsupported level.
The code also includes hooks for visual debugging (via \texttt{afwDisplay} and optional \texttt{matplotlib} plots), enabling visual inspection of PSF candidates and rejection reasons during development or QA.

\texttt{meas\_extensions\_psfex} is used in contexts where PSFEx compatibility is required or when modeling speed is preferred over robustness.
Another external PSF modeler (discussed in Section~\ref{sec:meas_extensions_piff}) provides greater robustness but is slower in comparison.
