\section{Instrument Signature Removal}
\label{sec:isr}

\subsection{Amplifier Offset Correction}
\label{sec:isr:ampoffset}
As part of the instrument signature removal (ISR) process, the amplifier offset
correction algorithm addresses systematic discontinuities in background sky
levels across amplifier boundaries. These discontinuities arise from electronic
biases between adjacent amplifiers and persist even after applying dark and
flat corrections. Drawing on the \texttt{PANSTARRS}' Pattern Continuity
algorithm \citep{2020ApJS..251....4W}, our method reduces these offsets,
thereby mitigating problems such as background oversubtraction and
undersubtraction at the boundaries.

The algorithm starts by computing differences between narrow stripes of
exposure along each amplifier edge and those of adjacent amplifiers,
considering only the unmasked regions. These differences are captured in a
matrix, where diagonal entries represent the number of neighboring amplifiers,
and off-diagonal entries reflect associations between the amplifiers. A second
matrix encodes directional information for these associations.
The amplifier offsets are adjusted by solving a matrix system derived from the
aforementioned matrices. This system is solved using least-squares minimization
to align neighboring amplifiers and smooth out discontinuities. The method is
generalized to support 2D amplifier layouts on a detector, incorporating
length-based weighting into the matrices to account for amplifiers that are not
necessarily square.