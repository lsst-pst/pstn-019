\section{Instrument Signature Removal}
\label{sec:isr}

\subsection{Amplifier Offset Correction}
\label{sec:isr:ampoffset}
The amplifier offset correction (commonly referred to as amp-offset correction, or pattern continuity correction) runs as part of the instrument signature removal (ISR) process.
This correction is designed to address systematic discontinuities in background sky levels across amplifier boundaries.
We believe that these discontinuities arise from electronic biases between adjacent amplifiers, persisting even after application of dark and flat corrections.

Drawing on the \texttt{PANSTARRS}' Pattern Continuity algorithm \citep{2020ApJS..251....4W}, our method aims to eliminate these offsets, thereby preventing problems such as background over-/under-subtraction at amplifier boundaries caused by discontinuities across the detector.

The amp-offset algorithm initially computes a robust flux difference measure between two narrow strips on opposite sides of each amplifier-amplifier interface.
Regions containing detected sources, or pixel data which have been masked for other reasons, are not considered.
These amp-interface differences are stored in an amp-offset matrix; diagonal entries represent the number of neighboring amplifiers, and off-diagonal entries encode information about the associations between amplifiers.
A complementary interface matrix encodes directional information for these associations.
Using this information, a least-squares minimization is performed to determine the optimal pedestal value to be added or subtracted to each amp which would reduce the amp-offset between that amplifier and all of its neighboring amplifiers.
This method is generalized to support 2D amplifier geometries within a detector, as with LSSTCam, incorporating length-based weighting into the matrices to account for amplifiers that are not square.